
% Default to the notebook output style

    


% Inherit from the specified cell style.




    
\documentclass[11pt]{article}

    
    
    \usepackage[T1]{fontenc}
    % Nicer default font (+ math font) than Computer Modern for most use cases
    \usepackage{mathpazo}

    % Basic figure setup, for now with no caption control since it's done
    % automatically by Pandoc (which extracts ![](path) syntax from Markdown).
    \usepackage{graphicx}
    % We will generate all images so they have a width \maxwidth. This means
    % that they will get their normal width if they fit onto the page, but
    % are scaled down if they would overflow the margins.
    \makeatletter
    \def\maxwidth{\ifdim\Gin@nat@width>\linewidth\linewidth
    \else\Gin@nat@width\fi}
    \makeatother
    \let\Oldincludegraphics\includegraphics
    % Set max figure width to be 80% of text width, for now hardcoded.
    \renewcommand{\includegraphics}[1]{\Oldincludegraphics[width=.8\maxwidth]{#1}}
    % Ensure that by default, figures have no caption (until we provide a
    % proper Figure object with a Caption API and a way to capture that
    % in the conversion process - todo).
    \usepackage{caption}
    \DeclareCaptionLabelFormat{nolabel}{}
    \captionsetup{labelformat=nolabel}

    \usepackage{adjustbox} % Used to constrain images to a maximum size 
    \usepackage{xcolor} % Allow colors to be defined
    \usepackage{enumerate} % Needed for markdown enumerations to work
    \usepackage{geometry} % Used to adjust the document margins
    \usepackage{amsmath} % Equations
    \usepackage{amssymb} % Equations
    \usepackage{textcomp} % defines textquotesingle
    % Hack from http://tex.stackexchange.com/a/47451/13684:
    \AtBeginDocument{%
        \def\PYZsq{\textquotesingle}% Upright quotes in Pygmentized code
    }
    \usepackage{upquote} % Upright quotes for verbatim code
    \usepackage{eurosym} % defines \euro
    \usepackage[mathletters]{ucs} % Extended unicode (utf-8) support
    \usepackage[utf8x]{inputenc} % Allow utf-8 characters in the tex document
    \usepackage{fancyvrb} % verbatim replacement that allows latex
    \usepackage{grffile} % extends the file name processing of package graphics 
                         % to support a larger range 
    % The hyperref package gives us a pdf with properly built
    % internal navigation ('pdf bookmarks' for the table of contents,
    % internal cross-reference links, web links for URLs, etc.)
    \usepackage{hyperref}
    \usepackage{longtable} % longtable support required by pandoc >1.10
    \usepackage{booktabs}  % table support for pandoc > 1.12.2
    \usepackage[inline]{enumitem} % IRkernel/repr support (it uses the enumerate* environment)
    \usepackage[normalem]{ulem} % ulem is needed to support strikethroughs (\sout)
                                % normalem makes italics be italics, not underlines
    

    
    
    % Colors for the hyperref package
    \definecolor{urlcolor}{rgb}{0,.145,.698}
    \definecolor{linkcolor}{rgb}{.71,0.21,0.01}
    \definecolor{citecolor}{rgb}{.12,.54,.11}

    % ANSI colors
    \definecolor{ansi-black}{HTML}{3E424D}
    \definecolor{ansi-black-intense}{HTML}{282C36}
    \definecolor{ansi-red}{HTML}{E75C58}
    \definecolor{ansi-red-intense}{HTML}{B22B31}
    \definecolor{ansi-green}{HTML}{00A250}
    \definecolor{ansi-green-intense}{HTML}{007427}
    \definecolor{ansi-yellow}{HTML}{DDB62B}
    \definecolor{ansi-yellow-intense}{HTML}{B27D12}
    \definecolor{ansi-blue}{HTML}{208FFB}
    \definecolor{ansi-blue-intense}{HTML}{0065CA}
    \definecolor{ansi-magenta}{HTML}{D160C4}
    \definecolor{ansi-magenta-intense}{HTML}{A03196}
    \definecolor{ansi-cyan}{HTML}{60C6C8}
    \definecolor{ansi-cyan-intense}{HTML}{258F8F}
    \definecolor{ansi-white}{HTML}{C5C1B4}
    \definecolor{ansi-white-intense}{HTML}{A1A6B2}

    % commands and environments needed by pandoc snippets
    % extracted from the output of `pandoc -s`
    \providecommand{\tightlist}{%
      \setlength{\itemsep}{0pt}\setlength{\parskip}{0pt}}
    \DefineVerbatimEnvironment{Highlighting}{Verbatim}{commandchars=\\\{\}}
    % Add ',fontsize=\small' for more characters per line
    \newenvironment{Shaded}{}{}
    \newcommand{\KeywordTok}[1]{\textcolor[rgb]{0.00,0.44,0.13}{\textbf{{#1}}}}
    \newcommand{\DataTypeTok}[1]{\textcolor[rgb]{0.56,0.13,0.00}{{#1}}}
    \newcommand{\DecValTok}[1]{\textcolor[rgb]{0.25,0.63,0.44}{{#1}}}
    \newcommand{\BaseNTok}[1]{\textcolor[rgb]{0.25,0.63,0.44}{{#1}}}
    \newcommand{\FloatTok}[1]{\textcolor[rgb]{0.25,0.63,0.44}{{#1}}}
    \newcommand{\CharTok}[1]{\textcolor[rgb]{0.25,0.44,0.63}{{#1}}}
    \newcommand{\StringTok}[1]{\textcolor[rgb]{0.25,0.44,0.63}{{#1}}}
    \newcommand{\CommentTok}[1]{\textcolor[rgb]{0.38,0.63,0.69}{\textit{{#1}}}}
    \newcommand{\OtherTok}[1]{\textcolor[rgb]{0.00,0.44,0.13}{{#1}}}
    \newcommand{\AlertTok}[1]{\textcolor[rgb]{1.00,0.00,0.00}{\textbf{{#1}}}}
    \newcommand{\FunctionTok}[1]{\textcolor[rgb]{0.02,0.16,0.49}{{#1}}}
    \newcommand{\RegionMarkerTok}[1]{{#1}}
    \newcommand{\ErrorTok}[1]{\textcolor[rgb]{1.00,0.00,0.00}{\textbf{{#1}}}}
    \newcommand{\NormalTok}[1]{{#1}}
    
    % Additional commands for more recent versions of Pandoc
    \newcommand{\ConstantTok}[1]{\textcolor[rgb]{0.53,0.00,0.00}{{#1}}}
    \newcommand{\SpecialCharTok}[1]{\textcolor[rgb]{0.25,0.44,0.63}{{#1}}}
    \newcommand{\VerbatimStringTok}[1]{\textcolor[rgb]{0.25,0.44,0.63}{{#1}}}
    \newcommand{\SpecialStringTok}[1]{\textcolor[rgb]{0.73,0.40,0.53}{{#1}}}
    \newcommand{\ImportTok}[1]{{#1}}
    \newcommand{\DocumentationTok}[1]{\textcolor[rgb]{0.73,0.13,0.13}{\textit{{#1}}}}
    \newcommand{\AnnotationTok}[1]{\textcolor[rgb]{0.38,0.63,0.69}{\textbf{\textit{{#1}}}}}
    \newcommand{\CommentVarTok}[1]{\textcolor[rgb]{0.38,0.63,0.69}{\textbf{\textit{{#1}}}}}
    \newcommand{\VariableTok}[1]{\textcolor[rgb]{0.10,0.09,0.49}{{#1}}}
    \newcommand{\ControlFlowTok}[1]{\textcolor[rgb]{0.00,0.44,0.13}{\textbf{{#1}}}}
    \newcommand{\OperatorTok}[1]{\textcolor[rgb]{0.40,0.40,0.40}{{#1}}}
    \newcommand{\BuiltInTok}[1]{{#1}}
    \newcommand{\ExtensionTok}[1]{{#1}}
    \newcommand{\PreprocessorTok}[1]{\textcolor[rgb]{0.74,0.48,0.00}{{#1}}}
    \newcommand{\AttributeTok}[1]{\textcolor[rgb]{0.49,0.56,0.16}{{#1}}}
    \newcommand{\InformationTok}[1]{\textcolor[rgb]{0.38,0.63,0.69}{\textbf{\textit{{#1}}}}}
    \newcommand{\WarningTok}[1]{\textcolor[rgb]{0.38,0.63,0.69}{\textbf{\textit{{#1}}}}}
    
    
    % Define a nice break command that doesn't care if a line doesn't already
    % exist.
    \def\br{\hspace*{\fill} \\* }
    % Math Jax compatability definitions
    \def\gt{>}
    \def\lt{<}
    % Document parameters
    \title{MTH5001FinalReport}
    
    
    

    % Pygments definitions
    
\makeatletter
\def\PY@reset{\let\PY@it=\relax \let\PY@bf=\relax%
    \let\PY@ul=\relax \let\PY@tc=\relax%
    \let\PY@bc=\relax \let\PY@ff=\relax}
\def\PY@tok#1{\csname PY@tok@#1\endcsname}
\def\PY@toks#1+{\ifx\relax#1\empty\else%
    \PY@tok{#1}\expandafter\PY@toks\fi}
\def\PY@do#1{\PY@bc{\PY@tc{\PY@ul{%
    \PY@it{\PY@bf{\PY@ff{#1}}}}}}}
\def\PY#1#2{\PY@reset\PY@toks#1+\relax+\PY@do{#2}}

\expandafter\def\csname PY@tok@w\endcsname{\def\PY@tc##1{\textcolor[rgb]{0.73,0.73,0.73}{##1}}}
\expandafter\def\csname PY@tok@c\endcsname{\let\PY@it=\textit\def\PY@tc##1{\textcolor[rgb]{0.25,0.50,0.50}{##1}}}
\expandafter\def\csname PY@tok@cp\endcsname{\def\PY@tc##1{\textcolor[rgb]{0.74,0.48,0.00}{##1}}}
\expandafter\def\csname PY@tok@k\endcsname{\let\PY@bf=\textbf\def\PY@tc##1{\textcolor[rgb]{0.00,0.50,0.00}{##1}}}
\expandafter\def\csname PY@tok@kp\endcsname{\def\PY@tc##1{\textcolor[rgb]{0.00,0.50,0.00}{##1}}}
\expandafter\def\csname PY@tok@kt\endcsname{\def\PY@tc##1{\textcolor[rgb]{0.69,0.00,0.25}{##1}}}
\expandafter\def\csname PY@tok@o\endcsname{\def\PY@tc##1{\textcolor[rgb]{0.40,0.40,0.40}{##1}}}
\expandafter\def\csname PY@tok@ow\endcsname{\let\PY@bf=\textbf\def\PY@tc##1{\textcolor[rgb]{0.67,0.13,1.00}{##1}}}
\expandafter\def\csname PY@tok@nb\endcsname{\def\PY@tc##1{\textcolor[rgb]{0.00,0.50,0.00}{##1}}}
\expandafter\def\csname PY@tok@nf\endcsname{\def\PY@tc##1{\textcolor[rgb]{0.00,0.00,1.00}{##1}}}
\expandafter\def\csname PY@tok@nc\endcsname{\let\PY@bf=\textbf\def\PY@tc##1{\textcolor[rgb]{0.00,0.00,1.00}{##1}}}
\expandafter\def\csname PY@tok@nn\endcsname{\let\PY@bf=\textbf\def\PY@tc##1{\textcolor[rgb]{0.00,0.00,1.00}{##1}}}
\expandafter\def\csname PY@tok@ne\endcsname{\let\PY@bf=\textbf\def\PY@tc##1{\textcolor[rgb]{0.82,0.25,0.23}{##1}}}
\expandafter\def\csname PY@tok@nv\endcsname{\def\PY@tc##1{\textcolor[rgb]{0.10,0.09,0.49}{##1}}}
\expandafter\def\csname PY@tok@no\endcsname{\def\PY@tc##1{\textcolor[rgb]{0.53,0.00,0.00}{##1}}}
\expandafter\def\csname PY@tok@nl\endcsname{\def\PY@tc##1{\textcolor[rgb]{0.63,0.63,0.00}{##1}}}
\expandafter\def\csname PY@tok@ni\endcsname{\let\PY@bf=\textbf\def\PY@tc##1{\textcolor[rgb]{0.60,0.60,0.60}{##1}}}
\expandafter\def\csname PY@tok@na\endcsname{\def\PY@tc##1{\textcolor[rgb]{0.49,0.56,0.16}{##1}}}
\expandafter\def\csname PY@tok@nt\endcsname{\let\PY@bf=\textbf\def\PY@tc##1{\textcolor[rgb]{0.00,0.50,0.00}{##1}}}
\expandafter\def\csname PY@tok@nd\endcsname{\def\PY@tc##1{\textcolor[rgb]{0.67,0.13,1.00}{##1}}}
\expandafter\def\csname PY@tok@s\endcsname{\def\PY@tc##1{\textcolor[rgb]{0.73,0.13,0.13}{##1}}}
\expandafter\def\csname PY@tok@sd\endcsname{\let\PY@it=\textit\def\PY@tc##1{\textcolor[rgb]{0.73,0.13,0.13}{##1}}}
\expandafter\def\csname PY@tok@si\endcsname{\let\PY@bf=\textbf\def\PY@tc##1{\textcolor[rgb]{0.73,0.40,0.53}{##1}}}
\expandafter\def\csname PY@tok@se\endcsname{\let\PY@bf=\textbf\def\PY@tc##1{\textcolor[rgb]{0.73,0.40,0.13}{##1}}}
\expandafter\def\csname PY@tok@sr\endcsname{\def\PY@tc##1{\textcolor[rgb]{0.73,0.40,0.53}{##1}}}
\expandafter\def\csname PY@tok@ss\endcsname{\def\PY@tc##1{\textcolor[rgb]{0.10,0.09,0.49}{##1}}}
\expandafter\def\csname PY@tok@sx\endcsname{\def\PY@tc##1{\textcolor[rgb]{0.00,0.50,0.00}{##1}}}
\expandafter\def\csname PY@tok@m\endcsname{\def\PY@tc##1{\textcolor[rgb]{0.40,0.40,0.40}{##1}}}
\expandafter\def\csname PY@tok@gh\endcsname{\let\PY@bf=\textbf\def\PY@tc##1{\textcolor[rgb]{0.00,0.00,0.50}{##1}}}
\expandafter\def\csname PY@tok@gu\endcsname{\let\PY@bf=\textbf\def\PY@tc##1{\textcolor[rgb]{0.50,0.00,0.50}{##1}}}
\expandafter\def\csname PY@tok@gd\endcsname{\def\PY@tc##1{\textcolor[rgb]{0.63,0.00,0.00}{##1}}}
\expandafter\def\csname PY@tok@gi\endcsname{\def\PY@tc##1{\textcolor[rgb]{0.00,0.63,0.00}{##1}}}
\expandafter\def\csname PY@tok@gr\endcsname{\def\PY@tc##1{\textcolor[rgb]{1.00,0.00,0.00}{##1}}}
\expandafter\def\csname PY@tok@ge\endcsname{\let\PY@it=\textit}
\expandafter\def\csname PY@tok@gs\endcsname{\let\PY@bf=\textbf}
\expandafter\def\csname PY@tok@gp\endcsname{\let\PY@bf=\textbf\def\PY@tc##1{\textcolor[rgb]{0.00,0.00,0.50}{##1}}}
\expandafter\def\csname PY@tok@go\endcsname{\def\PY@tc##1{\textcolor[rgb]{0.53,0.53,0.53}{##1}}}
\expandafter\def\csname PY@tok@gt\endcsname{\def\PY@tc##1{\textcolor[rgb]{0.00,0.27,0.87}{##1}}}
\expandafter\def\csname PY@tok@err\endcsname{\def\PY@bc##1{\setlength{\fboxsep}{0pt}\fcolorbox[rgb]{1.00,0.00,0.00}{1,1,1}{\strut ##1}}}
\expandafter\def\csname PY@tok@kc\endcsname{\let\PY@bf=\textbf\def\PY@tc##1{\textcolor[rgb]{0.00,0.50,0.00}{##1}}}
\expandafter\def\csname PY@tok@kd\endcsname{\let\PY@bf=\textbf\def\PY@tc##1{\textcolor[rgb]{0.00,0.50,0.00}{##1}}}
\expandafter\def\csname PY@tok@kn\endcsname{\let\PY@bf=\textbf\def\PY@tc##1{\textcolor[rgb]{0.00,0.50,0.00}{##1}}}
\expandafter\def\csname PY@tok@kr\endcsname{\let\PY@bf=\textbf\def\PY@tc##1{\textcolor[rgb]{0.00,0.50,0.00}{##1}}}
\expandafter\def\csname PY@tok@bp\endcsname{\def\PY@tc##1{\textcolor[rgb]{0.00,0.50,0.00}{##1}}}
\expandafter\def\csname PY@tok@fm\endcsname{\def\PY@tc##1{\textcolor[rgb]{0.00,0.00,1.00}{##1}}}
\expandafter\def\csname PY@tok@vc\endcsname{\def\PY@tc##1{\textcolor[rgb]{0.10,0.09,0.49}{##1}}}
\expandafter\def\csname PY@tok@vg\endcsname{\def\PY@tc##1{\textcolor[rgb]{0.10,0.09,0.49}{##1}}}
\expandafter\def\csname PY@tok@vi\endcsname{\def\PY@tc##1{\textcolor[rgb]{0.10,0.09,0.49}{##1}}}
\expandafter\def\csname PY@tok@vm\endcsname{\def\PY@tc##1{\textcolor[rgb]{0.10,0.09,0.49}{##1}}}
\expandafter\def\csname PY@tok@sa\endcsname{\def\PY@tc##1{\textcolor[rgb]{0.73,0.13,0.13}{##1}}}
\expandafter\def\csname PY@tok@sb\endcsname{\def\PY@tc##1{\textcolor[rgb]{0.73,0.13,0.13}{##1}}}
\expandafter\def\csname PY@tok@sc\endcsname{\def\PY@tc##1{\textcolor[rgb]{0.73,0.13,0.13}{##1}}}
\expandafter\def\csname PY@tok@dl\endcsname{\def\PY@tc##1{\textcolor[rgb]{0.73,0.13,0.13}{##1}}}
\expandafter\def\csname PY@tok@s2\endcsname{\def\PY@tc##1{\textcolor[rgb]{0.73,0.13,0.13}{##1}}}
\expandafter\def\csname PY@tok@sh\endcsname{\def\PY@tc##1{\textcolor[rgb]{0.73,0.13,0.13}{##1}}}
\expandafter\def\csname PY@tok@s1\endcsname{\def\PY@tc##1{\textcolor[rgb]{0.73,0.13,0.13}{##1}}}
\expandafter\def\csname PY@tok@mb\endcsname{\def\PY@tc##1{\textcolor[rgb]{0.40,0.40,0.40}{##1}}}
\expandafter\def\csname PY@tok@mf\endcsname{\def\PY@tc##1{\textcolor[rgb]{0.40,0.40,0.40}{##1}}}
\expandafter\def\csname PY@tok@mh\endcsname{\def\PY@tc##1{\textcolor[rgb]{0.40,0.40,0.40}{##1}}}
\expandafter\def\csname PY@tok@mi\endcsname{\def\PY@tc##1{\textcolor[rgb]{0.40,0.40,0.40}{##1}}}
\expandafter\def\csname PY@tok@il\endcsname{\def\PY@tc##1{\textcolor[rgb]{0.40,0.40,0.40}{##1}}}
\expandafter\def\csname PY@tok@mo\endcsname{\def\PY@tc##1{\textcolor[rgb]{0.40,0.40,0.40}{##1}}}
\expandafter\def\csname PY@tok@ch\endcsname{\let\PY@it=\textit\def\PY@tc##1{\textcolor[rgb]{0.25,0.50,0.50}{##1}}}
\expandafter\def\csname PY@tok@cm\endcsname{\let\PY@it=\textit\def\PY@tc##1{\textcolor[rgb]{0.25,0.50,0.50}{##1}}}
\expandafter\def\csname PY@tok@cpf\endcsname{\let\PY@it=\textit\def\PY@tc##1{\textcolor[rgb]{0.25,0.50,0.50}{##1}}}
\expandafter\def\csname PY@tok@c1\endcsname{\let\PY@it=\textit\def\PY@tc##1{\textcolor[rgb]{0.25,0.50,0.50}{##1}}}
\expandafter\def\csname PY@tok@cs\endcsname{\let\PY@it=\textit\def\PY@tc##1{\textcolor[rgb]{0.25,0.50,0.50}{##1}}}

\def\PYZbs{\char`\\}
\def\PYZus{\char`\_}
\def\PYZob{\char`\{}
\def\PYZcb{\char`\}}
\def\PYZca{\char`\^}
\def\PYZam{\char`\&}
\def\PYZlt{\char`\<}
\def\PYZgt{\char`\>}
\def\PYZsh{\char`\#}
\def\PYZpc{\char`\%}
\def\PYZdl{\char`\$}
\def\PYZhy{\char`\-}
\def\PYZsq{\char`\'}
\def\PYZdq{\char`\"}
\def\PYZti{\char`\~}
% for compatibility with earlier versions
\def\PYZat{@}
\def\PYZlb{[}
\def\PYZrb{]}
\makeatother


    % Exact colors from NB
    \definecolor{incolor}{rgb}{0.0, 0.0, 0.5}
    \definecolor{outcolor}{rgb}{0.545, 0.0, 0.0}



    
    % Prevent overflowing lines due to hard-to-break entities
    \sloppy 
    % Setup hyperref package
    \hypersetup{
      breaklinks=true,  % so long urls are correctly broken across lines
      colorlinks=true,
      urlcolor=urlcolor,
      linkcolor=linkcolor,
      citecolor=citecolor,
      }
    % Slightly bigger margins than the latex defaults
    
    \geometry{verbose,tmargin=1in,bmargin=1in,lmargin=1in,rmargin=1in}
    
    

    \begin{document}
    
    
    \maketitle
    
    

    
    \section{MTH5001: Introduction to Computer Programming
2018/19}\label{mth5001-introduction-to-computer-programming-201819}

\subsection{Final Report Project:
"Networks"}\label{final-report-project-networks}

    \subsubsection{Instructions:}\label{instructions}

First, please type your name and student number into the Markdown cell
below:

    \textbf{Name:}

\textbf{Student number:}

    You must write your answers in this Jupyter Notebook, using either
Markdown or Python code as appropriate. (You should create new code
and/or Markdown cells in the appropriate places, so that your answers
are clearly visible.)

Your code must be well documented. As a rough guide, you should aim to
include one line of comments for each line of code (but you may include
more or fewer comments depending on the situation). You should also use
sensible variable names, so that your code is as clear as possible. If
your code works but is unduly difficult to read, then you may lose
marks.

For this project, you will need to use the Python package
\href{https://networkx.github.io/}{NetworkX} extensively. However, to
test your coding skills, in certain questions you will be restricted to
using only specific functions. These restrictions are made clear below
(see questions 4 and 8).

\subsubsection{Submission deadline:}\label{submission-deadline}

You must submit your work via QMPlus (to the "Final Report Project"
assignment in the "Final Report Project" section).

The submission deadline is \textbf{11:55pm on Monday 29 April, 2019}.
Late submissions will be penalised according to the School's
\href{https://qmplus.qmul.ac.uk/mod/book/view.php?id=807735\&chapterid=89105}{guidelines}.

Your lecturers will respond to project-related emails until 5:00pm on
Friday 26 April, 2019, only. You should aim to have your project
finished by this time.

\subsubsection{Marking:}\label{marking}

The project is worth 70\% of your final mark for this module.

The total number of marks available for the project is 100.

Attempt all parts of all questions.

When writing up your project, good writing style is even more important
than in written exams. According to the
\href{https://qmplus.qmul.ac.uk/mod/book/view.php?id=807735\&chapterid=87786}{advice
in the student handbook},

\begin{quote}
To get full marks in any assessed work (tests or exams) you must
normally not only give the right answers but also explain your working
clearly and give reasons for your answers by writing legible and
grammatically correct English sentences. Mathematics is about logic and
reasoned arguments and the only way to present a reasoned and logical
argument is by writing about it clearly. Your writing may include
numbers and other mathematical symbols, but they are not enough on their
own. You should copy the writing style used in good mathematical
textbooks, such as those recommended for your modules. \textbf{You can
expect to lose marks for poor writing (incorrect grammar and spelling)
as well as for poor mathematics (incorrect or unclear logic).}
\end{quote}

\subsubsection{Plagiarism warning:}\label{plagiarism-warning}

Your work will be tested for plagiarism, which is an assessment offence,
according to the
\href{https://qmplus.qmul.ac.uk/mod/book/view.php?id=807735\&chapterid=87787}{School's
policy on Plagiarism}. In particular, while only academic staff will
make a judgement on whether plagiarism has occurred in a piece of work,
we will use the plagiarism detection software "Turnitin" to help us
assess how much of work matches other sources. You will have the
opportunity to upload your work, see the Turnitin result, and edit your
work accordingly before finalising your submission.

You may summarise relevant parts of books, online notes, or other
resources, as you see fit. However, you must use your own words as far
as possible (within reason, e.g. you would not be expected to change the
wording of a well-known theorem), and you \textbf{must}
\href{https://qmplus.qmul.ac.uk/mod/book/view.php?id=807735\&chapterid=87793}{reference}
any sources that you use. Similarly, if you decide to work with other
students on parts of the project, then you \textbf{must} write up your
work individually. You should also note that most of the questions are
personalised in the sense that you will need to import and manipulate
data that will be unique to you (i.e. no other student will have the
same data).

    \subsection{Background information}\label{background-information}

In this project you will learn about a field of mathematics called
\href{https://en.wikipedia.org/wiki/Graph_theory}{graph theory}. A
\textbf{graph} (or \textbf{network}) is simply a a collection of
\textbf{nodes} (or \textbf{vertices}), which may or may not be joined by
\textbf{edges}. (Note that this is not the same as the 'graph' of a
function.)

Graphs can represent all sorts of real-world (and, indeed, mathematical)
objects, e.g.

\begin{itemize}
\tightlist
\item
  social networks (nodes represent people, edges represent
  'friendship'),
\item
  molecules in chemistry/physics (nodes represent atoms, edges represent
  bonds),
\item
  communications networks, e.g. the internet (nodes represent
  computers/devices, edges represent connections).
\end{itemize}

In this project we will only consider \textbf{undirected} graphs (see
the above Wikipedia link for a definition).

Conventiently, Python has a package, called
\href{https://networkx.github.io/}{NetworkX}, for constructing and
analysing graphs. Let's look at an example. Below we create the famous
\href{https://en.wikipedia.org/wiki/Petersen_graph}{Petersen graph} and
use some basic NetworkX functions to learn a bit about it.

    \begin{Verbatim}[commandchars=\\\{\}]
{\color{incolor}In [{\color{incolor}2}]:} \PY{c+c1}{\PYZsh{} import NetworkX and other useful packages}
        \PY{k+kn}{import} \PY{n+nn}{numpy} \PY{k}{as} \PY{n+nn}{np}
        \PY{k+kn}{import} \PY{n+nn}{numpy}\PY{n+nn}{.}\PY{n+nn}{linalg} \PY{k}{as} \PY{n+nn}{la}
        \PY{k+kn}{import} \PY{n+nn}{matplotlib}\PY{n+nn}{.}\PY{n+nn}{pyplot} \PY{k}{as} \PY{n+nn}{plt}
        \PY{k+kn}{import} \PY{n+nn}{networkx} \PY{k}{as} \PY{n+nn}{nx}
        
        \PY{c+c1}{\PYZsh{} create the Petersen graph, storing it in a variable called \PYZdq{}PG\PYZdq{}}
        \PY{n}{PG} \PY{o}{=} \PY{n}{nx}\PY{o}{.}\PY{n}{petersen\PYZus{}graph}\PY{p}{(}\PY{p}{)}
\end{Verbatim}


    Before we doing anything else, it would make sense to draw the graph, to
get an idea of what it looks like. We can do this using the NetworkX
function \texttt{draw\_networkx} (together with our old favourtie,
matplotlib).

    \begin{Verbatim}[commandchars=\\\{\}]
{\color{incolor}In [{\color{incolor}2}]:} \PY{n}{nx}\PY{o}{.}\PY{n}{draw\PYZus{}networkx}\PY{p}{(}\PY{n}{PG}\PY{p}{,} \PY{n}{node\PYZus{}color} \PY{o}{=} \PY{l+s+s1}{\PYZsq{}}\PY{l+s+s1}{orange}\PY{l+s+s1}{\PYZsq{}}\PY{p}{,} \PY{n}{edge\PYZus{}color} \PY{o}{=} \PY{l+s+s1}{\PYZsq{}}\PY{l+s+s1}{blue}\PY{l+s+s1}{\PYZsq{}}\PY{p}{,} \PY{n}{with\PYZus{}labels}\PY{o}{=}\PY{k+kc}{True}\PY{p}{)}
        \PY{n}{plt}\PY{o}{.}\PY{n}{xticks}\PY{p}{(}\PY{p}{[}\PY{p}{]}\PY{p}{)}
        \PY{n}{plt}\PY{o}{.}\PY{n}{yticks}\PY{p}{(}\PY{p}{[}\PY{p}{]}\PY{p}{)}
        \PY{n}{plt}\PY{o}{.}\PY{n}{show}\PY{p}{(}\PY{p}{)}
\end{Verbatim}


    \begin{center}
    \adjustimage{max size={0.9\linewidth}{0.9\paperheight}}{output_7_0.png}
    \end{center}
    { \hspace*{\fill} \\}
    
    We can see that the graph has 10 nodes, labelled by the integers
\(0,1,\ldots,9\). It is also possible to label nodes with other data
types, most commonly strings, but we won't do that in this project. The
nodes of a graph can be accessed via the method \texttt{nodes()}:

    \begin{Verbatim}[commandchars=\\\{\}]
{\color{incolor}In [{\color{incolor}3}]:} \PY{n}{PG}\PY{o}{.}\PY{n}{nodes}\PY{p}{(}\PY{p}{)}
\end{Verbatim}


\begin{Verbatim}[commandchars=\\\{\}]
{\color{outcolor}Out[{\color{outcolor}3}]:} NodeView((0, 1, 2, 3, 4, 5, 6, 7, 8, 9))
\end{Verbatim}
            
    You can convert this to a Python list if you need to:

    \begin{Verbatim}[commandchars=\\\{\}]
{\color{incolor}In [{\color{incolor}4}]:} \PY{n+nb}{list}\PY{p}{(}\PY{n}{PG}\PY{o}{.}\PY{n}{nodes}\PY{p}{(}\PY{p}{)}\PY{p}{)}
\end{Verbatim}


\begin{Verbatim}[commandchars=\\\{\}]
{\color{outcolor}Out[{\color{outcolor}4}]:} [0, 1, 2, 3, 4, 5, 6, 7, 8, 9]
\end{Verbatim}
            
    This (hopefully) makes it clear that the node labels do in fact have
type \texttt{int}, at least in our example. You can also see from the
picture that the graph has 15 edges. These can be accessed using the
method \texttt{edges()}:

    \begin{Verbatim}[commandchars=\\\{\}]
{\color{incolor}In [{\color{incolor}5}]:} \PY{n}{PG}\PY{o}{.}\PY{n}{edges}\PY{p}{(}\PY{p}{)}
\end{Verbatim}


\begin{Verbatim}[commandchars=\\\{\}]
{\color{outcolor}Out[{\color{outcolor}5}]:} EdgeView([(0, 1), (0, 4), (0, 5), (1, 2), (1, 6), (2, 3), (2, 7), (3, 4), (3, 8), (4, 9), (5, 7), (5, 8), (6, 8), (6, 9), (7, 9)])
\end{Verbatim}
            
    Again, you can convert this to a list if you need to (try it), and you
will see that the elements of the list are tuples. In either case, if
you compare the output with the picture, it should become clear what it
means, i.e. two nodes labelled \(i\) and \(j\) are joined by an edge if
and only if the pair \((i, j)\) appears in \texttt{PG.edges()}.

    So far we haven't said much about how graphs are related to
\textbf{mathematics}. It turns out that a graph can be completely
defined by its
\href{https://en.wikipedia.org/wiki/Adjacency_matrix}{adjacency matrix}.
This is simply a matrix \(A\) defined as follows:

\begin{itemize}
\tightlist
\item
  \(A\) has size \(n \times n\), where \(n\) is the number of nodes in
  the graph;
\item
  if the nodes labelled \(i\) and \(j\) form an edge, then the
  \((i,j)\)-entry of \(A\) is \(1\); if they don't form an edge, the
  \((i,j)\)-entry of \(A\) is \(0\).
\end{itemize}

This idea is the foundation of
\href{https://en.wikipedia.org/wiki/Algebraic_graph_theory}{algebraic
graph theory}, a field of mathematics used to study graphs by analysing
certain matrices.

Not surprisingly, you can compute the adjacency matrix of a graph using
an appropriate NetworkX function. Let's do this for the Petersen graph:

    \begin{Verbatim}[commandchars=\\\{\}]
{\color{incolor}In [{\color{incolor}6}]:} \PY{n}{A} \PY{o}{=} \PY{n}{nx}\PY{o}{.}\PY{n}{adjacency\PYZus{}matrix}\PY{p}{(}\PY{n}{PG}\PY{p}{)}
\end{Verbatim}


    Note that if you print this 'adjacency matrix' (try it), it doesn't
actually look much like a matrix. This is because it doesn't have type
\texttt{numpy.ndarray} like the matrices/arrays we've worked with in
class:

    \begin{Verbatim}[commandchars=\\\{\}]
{\color{incolor}In [{\color{incolor}7}]:} \PY{n+nb}{type}\PY{p}{(}\PY{n}{nx}\PY{o}{.}\PY{n}{adjacency\PYZus{}matrix}\PY{p}{(}\PY{n}{PG}\PY{p}{)}\PY{p}{)}
\end{Verbatim}


\begin{Verbatim}[commandchars=\\\{\}]
{\color{outcolor}Out[{\color{outcolor}7}]:} scipy.sparse.csr.csr\_matrix
\end{Verbatim}
            
    However, you can convert it to a \texttt{numpy.ndarray} by calling the
method \texttt{toarray()}:

    \begin{Verbatim}[commandchars=\\\{\}]
{\color{incolor}In [{\color{incolor}8}]:} \PY{n}{A} \PY{o}{=} \PY{n}{A}\PY{o}{.}\PY{n}{toarray}\PY{p}{(}\PY{p}{)}
\end{Verbatim}


    \begin{Verbatim}[commandchars=\\\{\}]
{\color{incolor}In [{\color{incolor}9}]:} \PY{n+nb}{type}\PY{p}{(}\PY{n}{A}\PY{p}{)}
\end{Verbatim}


\begin{Verbatim}[commandchars=\\\{\}]
{\color{outcolor}Out[{\color{outcolor}9}]:} numpy.ndarray
\end{Verbatim}
            
    After doing this, the adjacency matrix looks like you would expect, so
you can use all the usual \texttt{numpy.linalg} functions on it:

    \begin{Verbatim}[commandchars=\\\{\}]
{\color{incolor}In [{\color{incolor}10}]:} \PY{n+nb}{print}\PY{p}{(}\PY{n}{A}\PY{p}{)}
\end{Verbatim}


    \begin{Verbatim}[commandchars=\\\{\}]
[[0 1 0 0 1 1 0 0 0 0]
 [1 0 1 0 0 0 1 0 0 0]
 [0 1 0 1 0 0 0 1 0 0]
 [0 0 1 0 1 0 0 0 1 0]
 [1 0 0 1 0 0 0 0 0 1]
 [1 0 0 0 0 0 0 1 1 0]
 [0 1 0 0 0 0 0 0 1 1]
 [0 0 1 0 0 1 0 0 0 1]
 [0 0 0 1 0 1 1 0 0 0]
 [0 0 0 0 1 0 1 1 0 0]]

    \end{Verbatim}

    Make sure that you understand what all these \(0\)'s and \(1\)'s mean:
the \((i,j)\)-entry of the adjacency matrix is \(1\) if and only if the
edges labelled \(i\) and \(j\) form an edge in the graph; otherwise, it
is \(0\). For example (remembering that Python starts counting from
\(0\), not from \(1\)): the \((0,4)\) entry is \(1\), and in the picture
above we see that nodes \(0\) and \(4\) form an edge; on the other hand,
the \((1,7)\) entry is \(0\), and accordingly nodes \(1\) and \(7\)
don't form an edge.

    You will be working with matrices related to graphs quite a lot in this
project, so before you begin you should make sure that you understand
what the code we've given you above is doing. You may also like to work
through the official
\href{https://networkx.github.io/documentation/stable/tutorial.html}{NetworkX
tutorial} before attempting the project, bearing in mind that not
everything in the tutorial is relevant to the project. (Alternatively,
Google for another tutorial if you don't like that one.)

\textbf{A final remark before we get to the project itself:}

You can rest assured that the graphs we consider this project all have
the following nice properties:

\begin{itemize}
\tightlist
\item
  They are \textbf{connected}. This means that for every pair of nodes
  \(i\) and \(j\), there is a 'path' of edges joining \(i\) to \(j\).
  For example, the Petersen graph is connected, e.g. the nodes labelled
  \(6\) and \(7\) do not form an edge, but we can still reach node \(7\)
  from node \(6\) via the edges \((6,9)\) and \((9,7)\).
\item
  They are \textbf{simple}. This means that there is never an edge from
  a node to itself.
\end{itemize}

You may come across these terms when you are researching the relevant
mathematics for various parts of the project, so you should know what
they mean.

    \subsection{The project}\label{the-project}

As we have already mentioned, in this project you will make extensive
use of the Python package \href{https://networkx.github.io/}{NetworkX},
which allows you to create and analyse graphs. You are expected to read
the relevant parts of the NetworkX documentation, or otherwise learn how
to use whatever Python functions you need to complete the project.
However, the mini-tutorial which we have provided above should be enough
to get you started.

You will also need to research and summarise some mathematics related to
graphs, and to use your findings to write certain pieces of code 'from
scratch', instead of of using NetworkX functions. In these cases
(questions 4 and 8), it is \textbf{strongly recommended} that you use
NetworkX to check your answers.

You should structure your report as follows:

\subsubsection{Part I: Data import and preliminary investigation {[}10
marks{]}}\label{part-i-data-import-and-preliminary-investigation-10-marks}

You have been provided with a Python file called \textbf{"data.py"} on
QMPlus, which you should save in the same directory as this Jupyter
notebook. This file contains a function \texttt{create\_graph} which
constructs a random graph that you will be analysing throughout the
project. By following the instructions in question 1 (below), you will
create a graph that is unique to you, i.e. no two students will have the
same graph.

\textbf{1. {[}5 marks{]}} Execute the following code cell to create your
graph, storing it in a variable called \texttt{G} (you can change the
variable name if you like, but we recommend leaving it as it is). You
\textbf{must} replace the number "123456789" with your 9-digit student
number.

\emph{Important note: If you do not do this correctly, you will score 0
for this question, and if you are found to have used the same input as
another student (rather than your individual student number), then your
submission will be reviewed for plagiarism.}

    \begin{Verbatim}[commandchars=\\\{\}]
{\color{incolor}In [{\color{incolor}3}]:} \PY{k+kn}{from} \PY{n+nn}{data} \PY{k}{import} \PY{n}{create\PYZus{}graph}
        \PY{c+c1}{\PYZsh{} import NetworkX and other useful packages}
        \PY{k+kn}{import} \PY{n+nn}{numpy} \PY{k}{as} \PY{n+nn}{np}
        \PY{k+kn}{import} \PY{n+nn}{numpy}\PY{n+nn}{.}\PY{n+nn}{linalg} \PY{k}{as} \PY{n+nn}{la}
        \PY{k+kn}{import} \PY{n+nn}{matplotlib}\PY{n+nn}{.}\PY{n+nn}{pyplot} \PY{k}{as} \PY{n+nn}{plt}
        \PY{k+kn}{import} \PY{n+nn}{networkx} \PY{k}{as} \PY{n+nn}{nx}
        
        \PY{c+c1}{\PYZsh{} Replace \PYZdq{}123456789\PYZdq{} below with your 9\PYZhy{}digit student number.}
        \PY{c+c1}{\PYZsh{} My student number is \PYZdq{}160626247\PYZdq{}.}
        \PY{n}{G} \PY{o}{=} \PY{n}{create\PYZus{}graph}\PY{p}{(}\PY{l+m+mi}{160626247}\PY{p}{)}
\end{Verbatim}


    \textbf{2. {[}5 marks{]}} Draw your graph, and calculate how many nodes
and edges it has.

    \begin{Verbatim}[commandchars=\\\{\}]
{\color{incolor}In [{\color{incolor}188}]:} \PY{c+c1}{\PYZsh{} Draw the graph}
          \PY{n}{plt}\PY{o}{.}\PY{n}{figure}\PY{p}{(}\PY{n}{figsize}\PY{o}{=}\PY{p}{(}\PY{l+m+mi}{18}\PY{p}{,}\PY{l+m+mi}{18}\PY{p}{)}\PY{p}{)}
          \PY{n}{nx}\PY{o}{.}\PY{n}{draw\PYZus{}networkx}\PY{p}{(}\PY{n}{G}\PY{p}{,} \PY{n}{node\PYZus{}color} \PY{o}{=} \PY{l+s+s1}{\PYZsq{}}\PY{l+s+s1}{Coral}\PY{l+s+s1}{\PYZsq{}}\PY{p}{,} \PY{n}{edge\PYZus{}color} \PY{o}{=} \PY{l+s+s1}{\PYZsq{}}\PY{l+s+s1}{RoyalBlue}\PY{l+s+s1}{\PYZsq{}}\PY{p}{,} \PY{n}{with\PYZus{}labels}\PY{o}{=}\PY{k+kc}{True}\PY{p}{)}
          \PY{n}{plt}\PY{o}{.}\PY{n}{xticks}\PY{p}{(}\PY{p}{[}\PY{p}{]}\PY{p}{)}
          \PY{n}{plt}\PY{o}{.}\PY{n}{yticks}\PY{p}{(}\PY{p}{[}\PY{p}{]}\PY{p}{)}
          \PY{n}{plt}\PY{o}{.}\PY{n}{show}\PY{p}{(}\PY{p}{)}
\end{Verbatim}


    \begin{center}
    \adjustimage{max size={0.9\linewidth}{0.9\paperheight}}{output_29_0.png}
    \end{center}
    { \hspace*{\fill} \\}
    
    \begin{Verbatim}[commandchars=\\\{\}]
{\color{incolor}In [{\color{incolor}5}]:} \PY{n}{G}\PY{o}{.}\PY{n}{nodes}\PY{p}{(}\PY{p}{)}
\end{Verbatim}


\begin{Verbatim}[commandchars=\\\{\}]
{\color{outcolor}Out[{\color{outcolor}5}]:} NodeView((0, 1, 2, 3, 4, 5, 6, 7, 8, 9, 10, 11, 12, 13, 14, 15, 16, 17, 18, 19, 20, 21, 22, 23, 24, 25, 26, 27, 28, 29, 30, 31, 32, 33, 34, 35, 36, 37, 38, 39, 40, 41, 42, 43, 44, 45, 46, 47, 48))
\end{Verbatim}
            
    \begin{Verbatim}[commandchars=\\\{\}]
{\color{incolor}In [{\color{incolor}6}]:} \PY{n+nb}{list}\PY{p}{(}\PY{n}{G}\PY{o}{.}\PY{n}{nodes}\PY{p}{(}\PY{p}{)}\PY{p}{)}
\end{Verbatim}


\begin{Verbatim}[commandchars=\\\{\}]
{\color{outcolor}Out[{\color{outcolor}6}]:} [0,
         1,
         2,
         3,
         4,
         5,
         6,
         7,
         8,
         9,
         10,
         11,
         12,
         13,
         14,
         15,
         16,
         17,
         18,
         19,
         20,
         21,
         22,
         23,
         24,
         25,
         26,
         27,
         28,
         29,
         30,
         31,
         32,
         33,
         34,
         35,
         36,
         37,
         38,
         39,
         40,
         41,
         42,
         43,
         44,
         45,
         46,
         47,
         48]
\end{Verbatim}
            
    \begin{Verbatim}[commandchars=\\\{\}]
{\color{incolor}In [{\color{incolor}7}]:} \PY{c+c1}{\PYZsh{} Calculate number of nodes}
        \PY{n+nb}{print}\PY{p}{(}\PY{l+s+s2}{\PYZdq{}}\PY{l+s+s2}{number of nodes is: }\PY{l+s+s2}{\PYZdq{}}\PY{p}{,}\PY{n}{G}\PY{o}{.}\PY{n}{number\PYZus{}of\PYZus{}nodes}\PY{p}{(}\PY{p}{)}\PY{p}{)}
\end{Verbatim}


    \begin{Verbatim}[commandchars=\\\{\}]
number of nodes is:  49

    \end{Verbatim}

    \begin{Verbatim}[commandchars=\\\{\}]
{\color{incolor}In [{\color{incolor}8}]:} \PY{c+c1}{\PYZsh{} Calculate number of edges of this graph G.}
        \PY{n+nb}{print}\PY{p}{(}\PY{l+s+s2}{\PYZdq{}}\PY{l+s+s2}{number of edges is: }\PY{l+s+s2}{\PYZdq{}}\PY{p}{,}\PY{n}{G}\PY{o}{.}\PY{n}{number\PYZus{}of\PYZus{}edges}\PY{p}{(}\PY{p}{)}\PY{p}{)}
\end{Verbatim}


    \begin{Verbatim}[commandchars=\\\{\}]
number of edges is:  97

    \end{Verbatim}

    \subsubsection{Part II: Distance matrices and shortest paths {[}30
marks{]}}\label{part-ii-distance-matrices-and-shortest-paths-30-marks}

Many properties of graphs can be analysed by using matrices/linear
algebra. The rest of your report will involve researching/summarising
some of the relevant mathematics, and writing/using Python code to
analyse certain properties of your graph from Part I. As explained
above, you are allowed to summarise information from books and/or web
pages, but you must use your own words and clearly reference any sources
you use.

\textbf{3. {[}10 marks{]}} Explain what a "path" between two nodes in a
graph is, and what the "distance" between two nodes is. Explain also
what the "distance matrix" of a graph is, and how it can be computed
using the adjacency matrix. Here you should discuss arbitrary
(undirected, simple, connected) graphs, not your specific graph from
Part I.

Note: You do \textbf{not} need to give any proofs, but you must
reference any material you use, as explained in the plagiarism warning
above.

    \textbf{Answer:} - 3.1 path: - a path between two nodes in a graph: -
Generally, a path between two nodes in a graph could be considered as a
way to get to a node from another node, by passing through other nodes
and edges linked them together. - in undirected graphs, a path between
two nodes could be understood as a double-ended linked list of the nodes
connected by edges, which could link these two nodes together, where one
node is the start point and the other is the end point, and vice verse.
- in connected graphs, a path between two nodes could be understood as a
single direction linked list of nodes, connected by directed edges.

\begin{itemize}
\tightlist
\item
  3.2 distance matrix

  \begin{itemize}
  \tightlist
  \item
    In graph theory, a distance matrix is a two-dimensioanl array that
    contains the pairwise distances between any two connected nodes in a
    graph, which primary designed to show the distance between a pair of
    two directly connected nodes.
  \end{itemize}
\end{itemize}

    \textbf{4. {[}10 marks{]}} Write a function \texttt{distance\_matrix}
which computes the distance matrix of a graph. Your function should
return a matrix, represented as an array of type \texttt{numpy.ndarray},
of the same shape as the adjacency matrix of the graph. You may use the
NetworkX function \texttt{adjacency\_matrix} to compute the adjacency
matrix of the input graph, but you \textbf{must not use any other
NetworkX functions}.

    \begin{Verbatim}[commandchars=\\\{\}]
{\color{incolor}In [{\color{incolor}9}]:} \PY{k}{def} \PY{n+nf}{getMinDistance}\PY{p}{(}\PY{n}{dist}\PY{p}{,} \PY{n}{queue}\PY{p}{)}\PY{p}{:}
            \PY{c+c1}{\PYZsh{} Initialize the minimum value and its index}
            \PY{n}{min\PYZus{}val} \PY{o}{=} \PY{n+nb}{float}\PY{p}{(}\PY{l+s+s2}{\PYZdq{}}\PY{l+s+s2}{Inf}\PY{l+s+s2}{\PYZdq{}}\PY{p}{)}
            \PY{n}{min\PYZus{}idx} \PY{o}{=} \PY{o}{\PYZhy{}}\PY{l+m+mi}{1}
            \PY{n}{length} \PY{o}{=} \PY{n+nb}{len}\PY{p}{(}\PY{n}{dist}\PY{p}{)}
            \PY{c+c1}{\PYZsh{} get the minimum value from the distance array}
            \PY{k}{for} \PY{n}{i} \PY{o+ow}{in} \PY{n+nb}{range}\PY{p}{(}\PY{n}{length}\PY{p}{)}\PY{p}{:}
                \PY{c+c1}{\PYZsh{} update the min\PYZus{}val and min\PYZus{}idx}
                \PY{k}{if} \PY{p}{(}\PY{n}{dist}\PY{p}{[}\PY{n}{i}\PY{p}{]} \PY{o}{\PYZlt{}} \PY{n}{min\PYZus{}val} \PY{o+ow}{and} \PY{n}{i} \PY{o+ow}{in} \PY{n}{queue}\PY{p}{)}\PY{p}{:}
                    \PY{n}{min\PYZus{}val} \PY{o}{=} \PY{n}{dist}\PY{p}{[}\PY{n}{i}\PY{p}{]}
                    \PY{n}{min\PYZus{}idx} \PY{o}{=} \PY{n}{i}
            \PY{k}{return} \PY{n}{min\PYZus{}idx}
\end{Verbatim}


    \begin{Verbatim}[commandchars=\\\{\}]
{\color{incolor}In [{\color{incolor}10}]:} \PY{k}{def} \PY{n+nf}{dijkstraAlg}\PY{p}{(}\PY{n}{adj\PYZus{}matrix}\PY{p}{,} \PY{n}{src}\PY{p}{)}\PY{p}{:}
             \PY{n}{row} \PY{o}{=} \PY{n+nb}{len}\PY{p}{(}\PY{n}{adj\PYZus{}matrix}\PY{p}{)}
             \PY{n}{col} \PY{o}{=} \PY{n+nb}{len}\PY{p}{(}\PY{n}{adj\PYZus{}matrix}\PY{p}{[}\PY{l+m+mi}{0}\PY{p}{]}\PY{p}{)}
             
             \PY{c+c1}{\PYZsh{} distanc array: dist[i] is the shortest distance from the src node to i.}
             \PY{c+c1}{\PYZsh{} initialize the distances to be INFINTE.}
             \PY{n}{dist} \PY{o}{=} \PY{p}{[}\PY{n+nb}{float}\PY{p}{(}\PY{l+s+s2}{\PYZdq{}}\PY{l+s+s2}{Inf}\PY{l+s+s2}{\PYZdq{}}\PY{p}{)}\PY{p}{]} \PY{o}{*} \PY{n}{col}
             \PY{n}{parent} \PY{o}{=} \PY{p}{[}\PY{o}{\PYZhy{}}\PY{l+m+mi}{1}\PY{p}{]} \PY{o}{*} \PY{n}{col}
             
             \PY{c+c1}{\PYZsh{} the distance from the src node to itself is 0.}
             \PY{n}{dist}\PY{p}{[}\PY{n}{src}\PY{p}{]} \PY{o}{=} \PY{l+m+mi}{0}
             
             \PY{n}{queue} \PY{o}{=} \PY{p}{[}\PY{p}{]}
             \PY{k}{for} \PY{n}{i} \PY{o+ow}{in} \PY{n+nb}{range}\PY{p}{(}\PY{n}{col}\PY{p}{)}\PY{p}{:}
                 \PY{c+c1}{\PYZsh{} add all nodes in queue.}
                 \PY{n}{queue}\PY{o}{.}\PY{n}{append}\PY{p}{(}\PY{n}{i}\PY{p}{)} 
                 
             \PY{c+c1}{\PYZsh{} Find the shortest path for all the nodes in the graph.}
             \PY{k}{while} \PY{n}{queue}\PY{p}{:}
                 \PY{c+c1}{\PYZsh{} get index of the min\PYZus{}val from the nodes in the queue.}
                 \PY{n}{idx} \PY{o}{=} \PY{n}{getMinDistance}\PY{p}{(}\PY{n}{dist}\PY{p}{,} \PY{n}{queue}\PY{p}{)}
                 \PY{c+c1}{\PYZsh{} remove the min element from the queue.}
                 \PY{n}{queue}\PY{o}{.}\PY{n}{remove}\PY{p}{(}\PY{n}{idx}\PY{p}{)}
         
                 \PY{c+c1}{\PYZsh{} update}
                 \PY{k}{for} \PY{n}{i} \PY{o+ow}{in} \PY{n+nb}{range}\PY{p}{(}\PY{n}{col}\PY{p}{)}\PY{p}{:}
                     \PY{k}{if} \PY{p}{(}\PY{n}{adj\PYZus{}matrix}\PY{p}{[}\PY{n}{idx}\PY{p}{]}\PY{p}{[}\PY{n}{i}\PY{p}{]} \PY{o+ow}{and} \PY{n}{i} \PY{o+ow}{in} \PY{n}{queue}\PY{p}{)}\PY{p}{:}
                         \PY{k}{if} \PY{p}{(}\PY{n}{dist}\PY{p}{[}\PY{n}{idx}\PY{p}{]} \PY{o}{+} \PY{n}{adj\PYZus{}matrix}\PY{p}{[}\PY{n}{idx}\PY{p}{]}\PY{p}{[}\PY{n}{i}\PY{p}{]} \PY{o}{\PYZlt{}} \PY{n}{dist}\PY{p}{[}\PY{n}{i}\PY{p}{]}\PY{p}{)}\PY{p}{:}
                             \PY{n}{dist}\PY{p}{[}\PY{n}{i}\PY{p}{]} \PY{o}{=} \PY{n}{dist}\PY{p}{[}\PY{n}{idx}\PY{p}{]} \PY{o}{+} \PY{n}{adj\PYZus{}matrix}\PY{p}{[}\PY{n}{idx}\PY{p}{]}\PY{p}{[}\PY{n}{i}\PY{p}{]}
                             \PY{n}{parent}\PY{p}{[}\PY{n}{i}\PY{p}{]} \PY{o}{=} \PY{n}{idx}
             \PY{k}{return} \PY{n}{dist}            
                 
\end{Verbatim}


    \begin{Verbatim}[commandchars=\\\{\}]
{\color{incolor}In [{\color{incolor}11}]:} \PY{k}{def} \PY{n+nf}{computeDistMat}\PY{p}{(}\PY{n}{graph}\PY{p}{)}\PY{p}{:}
             \PY{n}{adj\PYZus{}matrix} \PY{o}{=} \PY{n}{nx}\PY{o}{.}\PY{n}{adjacency\PYZus{}matrix}\PY{p}{(}\PY{n}{graph}\PY{p}{)}
             \PY{n}{adj\PYZus{}matrix} \PY{o}{=} \PY{n}{adj\PYZus{}matrix}\PY{o}{.}\PY{n}{toarray}\PY{p}{(}\PY{p}{)}
             
             \PY{n}{row} \PY{o}{=} \PY{n+nb}{len}\PY{p}{(}\PY{n}{adj\PYZus{}matrix}\PY{p}{)}
             \PY{n}{col} \PY{o}{=} \PY{n+nb}{len}\PY{p}{(}\PY{n}{adj\PYZus{}matrix}\PY{p}{[}\PY{l+m+mi}{0}\PY{p}{]}\PY{p}{)}
             
             \PY{n}{dist\PYZus{}matrix} \PY{o}{=} \PY{p}{[}\PY{p}{[}\PY{n+nb}{float}\PY{p}{(}\PY{l+s+s2}{\PYZdq{}}\PY{l+s+s2}{Inf}\PY{l+s+s2}{\PYZdq{}}\PY{p}{)}\PY{p}{]} \PY{o}{*} \PY{n}{col}\PY{p}{]} \PY{o}{*} \PY{n}{row}
             
             \PY{k}{for} \PY{n}{i} \PY{o+ow}{in} \PY{n+nb}{range}\PY{p}{(}\PY{n}{row}\PY{p}{)}\PY{p}{:}
                 \PY{n}{row\PYZus{}dist} \PY{o}{=} \PY{n}{dijkstraAlg}\PY{p}{(}\PY{n}{adj\PYZus{}matrix}\PY{p}{,}\PY{n}{i}\PY{p}{)}
                 \PY{n}{dist\PYZus{}matrix}\PY{p}{[}\PY{n}{i}\PY{p}{]} \PY{o}{=} \PY{n}{row\PYZus{}dist}
             
             \PY{k}{return} \PY{n}{dist\PYZus{}matrix}
            
\end{Verbatim}


    \begin{Verbatim}[commandchars=\\\{\}]
{\color{incolor}In [{\color{incolor}12}]:} \PY{n}{dist\PYZus{}matrix} \PY{o}{=} \PY{n}{computeDistMat}\PY{p}{(}\PY{n}{G}\PY{p}{)}
         \PY{n+nb}{print}\PY{p}{(}\PY{n}{dist\PYZus{}matrix}\PY{p}{)}
\end{Verbatim}


    \begin{Verbatim}[commandchars=\\\{\}]
[[0, 1, 2, 2, 3, 2, 2, 2, 3, 3, 2, 3, 2, 2, 1, 3, 2, 3, 2, 2, 3, 1, 2, 2, 1, 2, 3, 2, 2, 3, 1, 3, 2, 3, 3, 2, 3, 2, 3, 3, 3, 2, 2, 2, 3, 1, 2, 3, 2], [1, 0, 3, 3, 4, 3, 3, 3, 2, 4, 3, 2, 3, 1, 2, 4, 1, 4, 3, 1, 2, 2, 3, 3, 2, 3, 2, 1, 3, 3, 2, 4, 2, 4, 2, 3, 2, 2, 3, 3, 3, 3, 1, 3, 3, 2, 3, 4, 3], [2, 3, 0, 3, 4, 2, 2, 2, 3, 3, 3, 3, 3, 2, 2, 4, 4, 4, 2, 4, 3, 1, 4, 3, 2, 2, 3, 4, 2, 2, 3, 4, 1, 3, 5, 4, 3, 1, 3, 1, 2, 1, 2, 3, 2, 3, 3, 4, 3], [2, 3, 3, 0, 4, 2, 3, 3, 2, 2, 3, 3, 2, 3, 1, 1, 4, 4, 2, 3, 3, 3, 2, 3, 3, 4, 1, 3, 3, 3, 3, 1, 2, 4, 2, 1, 3, 2, 2, 3, 4, 4, 2, 3, 1, 2, 2, 2, 2], [3, 4, 4, 4, 0, 4, 2, 3, 4, 4, 1, 2, 4, 5, 4, 5, 5, 5, 4, 5, 5, 3, 3, 4, 4, 4, 4, 5, 4, 3, 2, 4, 4, 3, 4, 3, 4, 4, 4, 5, 5, 4, 3, 4, 5, 3, 5, 2, 3], [2, 3, 2, 2, 4, 0, 2, 2, 2, 4, 3, 3, 1, 3, 2, 3, 4, 4, 1, 3, 2, 1, 3, 3, 2, 2, 1, 3, 2, 2, 3, 2, 2, 3, 4, 3, 3, 3, 1, 3, 3, 2, 2, 3, 3, 3, 3, 3, 2], [2, 3, 2, 3, 2, 2, 0, 2, 3, 3, 1, 2, 3, 3, 2, 4, 4, 3, 2, 4, 3, 1, 3, 2, 2, 2, 3, 4, 2, 1, 2, 3, 2, 1, 3, 2, 2, 2, 2, 3, 3, 2, 3, 3, 3, 1, 3, 2, 3], [2, 3, 2, 3, 3, 2, 2, 0, 1, 4, 2, 3, 3, 2, 3, 4, 4, 4, 2, 2, 4, 1, 2, 3, 2, 2, 2, 4, 2, 3, 1, 3, 2, 3, 4, 3, 3, 3, 3, 3, 3, 2, 3, 3, 3, 3, 4, 3, 2], [3, 2, 3, 2, 4, 2, 3, 1, 0, 3, 3, 3, 3, 1, 3, 3, 3, 5, 3, 1, 4, 2, 1, 4, 3, 3, 1, 3, 3, 3, 2, 2, 3, 4, 3, 2, 2, 2, 3, 4, 4, 3, 2, 4, 3, 3, 4, 3, 3], [3, 4, 3, 2, 4, 4, 3, 4, 3, 0, 3, 4, 3, 4, 3, 3, 5, 4, 3, 4, 4, 3, 2, 3, 3, 4, 3, 3, 4, 4, 3, 3, 2, 4, 2, 1, 3, 4, 4, 3, 5, 4, 3, 3, 1, 2, 4, 2, 2], [2, 3, 3, 3, 1, 3, 1, 2, 3, 3, 0, 1, 3, 4, 3, 4, 4, 4, 3, 4, 4, 2, 2, 3, 3, 3, 3, 4, 3, 2, 1, 3, 3, 2, 3, 2, 3, 3, 3, 4, 4, 3, 2, 3, 4, 2, 4, 1, 2], [3, 2, 3, 3, 2, 3, 2, 3, 3, 4, 1, 0, 4, 3, 4, 4, 3, 5, 3, 3, 4, 3, 3, 4, 4, 4, 2, 3, 4, 3, 2, 4, 2, 3, 4, 3, 4, 4, 4, 3, 5, 4, 1, 4, 3, 3, 5, 2, 3], [2, 3, 3, 2, 4, 1, 3, 3, 3, 3, 3, 4, 0, 3, 1, 3, 4, 3, 2, 4, 2, 2, 3, 2, 2, 3, 2, 3, 1, 2, 2, 2, 3, 2, 3, 2, 2, 2, 1, 4, 3, 3, 3, 2, 3, 2, 2, 2, 1], [2, 1, 2, 3, 5, 3, 3, 2, 1, 4, 4, 3, 3, 0, 2, 4, 2, 5, 3, 2, 3, 3, 2, 4, 3, 4, 2, 2, 4, 2, 3, 3, 3, 4, 3, 3, 1, 1, 3, 3, 4, 3, 2, 3, 4, 3, 3, 3, 2], [1, 2, 2, 1, 4, 2, 2, 3, 3, 3, 3, 4, 1, 2, 0, 2, 3, 3, 1, 3, 3, 2, 3, 2, 2, 3, 2, 3, 2, 2, 2, 2, 2, 3, 3, 2, 3, 1, 2, 3, 4, 3, 3, 3, 2, 1, 1, 3, 2], [3, 4, 4, 1, 5, 3, 4, 4, 3, 3, 4, 4, 3, 4, 2, 0, 5, 5, 3, 4, 4, 4, 3, 4, 4, 5, 2, 4, 4, 4, 4, 2, 3, 5, 3, 2, 4, 3, 3, 4, 5, 5, 3, 4, 2, 3, 3, 3, 3], [2, 1, 4, 4, 5, 4, 4, 4, 3, 5, 4, 3, 4, 2, 3, 5, 0, 5, 4, 2, 3, 3, 4, 4, 3, 4, 3, 2, 4, 4, 3, 5, 3, 5, 3, 4, 3, 3, 4, 4, 4, 4, 2, 4, 4, 3, 4, 5, 4], [3, 4, 4, 4, 5, 4, 3, 4, 5, 4, 4, 5, 3, 5, 3, 5, 5, 0, 3, 5, 5, 3, 4, 1, 2, 4, 5, 5, 2, 4, 4, 5, 4, 3, 4, 3, 4, 4, 4, 5, 5, 4, 5, 3, 5, 2, 2, 4, 3], [2, 3, 2, 2, 4, 1, 2, 2, 3, 3, 3, 3, 2, 3, 1, 3, 4, 3, 0, 4, 3, 1, 4, 2, 1, 2, 2, 4, 2, 3, 3, 3, 1, 3, 4, 3, 3, 2, 2, 2, 3, 2, 2, 2, 2, 2, 2, 3, 2], [2, 1, 4, 3, 5, 3, 4, 2, 1, 4, 4, 3, 4, 2, 3, 4, 2, 5, 4, 0, 3, 3, 2, 4, 3, 4, 2, 2, 4, 4, 3, 3, 3, 5, 3, 3, 3, 3, 4, 4, 4, 4, 2, 4, 4, 3, 4, 4, 4], [3, 2, 3, 3, 5, 2, 3, 4, 4, 4, 4, 4, 2, 3, 3, 4, 3, 5, 3, 3, 0, 3, 3, 4, 4, 4, 3, 1, 3, 2, 4, 2, 4, 4, 2, 3, 3, 3, 1, 4, 1, 2, 3, 4, 4, 4, 4, 4, 3], [1, 2, 1, 3, 3, 1, 1, 1, 2, 3, 2, 3, 2, 3, 2, 4, 3, 3, 1, 3, 3, 0, 3, 2, 1, 1, 2, 3, 1, 2, 2, 3, 1, 2, 4, 3, 3, 2, 2, 2, 2, 1, 2, 2, 2, 2, 3, 3, 2], [2, 3, 4, 2, 3, 3, 3, 2, 1, 2, 2, 3, 3, 2, 3, 3, 4, 4, 4, 2, 3, 3, 0, 3, 3, 4, 2, 3, 4, 3, 1, 1, 4, 4, 2, 1, 3, 3, 2, 5, 4, 4, 3, 3, 3, 2, 4, 2, 2], [2, 3, 3, 3, 4, 3, 2, 3, 4, 3, 3, 4, 2, 4, 2, 4, 4, 1, 2, 4, 4, 2, 3, 0, 1, 3, 4, 4, 1, 3, 3, 4, 3, 2, 3, 2, 3, 3, 3, 4, 4, 3, 4, 2, 4, 1, 1, 3, 2], [1, 2, 2, 3, 4, 2, 2, 2, 3, 3, 3, 4, 2, 3, 2, 4, 3, 2, 1, 3, 4, 1, 3, 1, 0, 2, 3, 3, 1, 3, 2, 4, 2, 2, 3, 2, 2, 3, 3, 3, 3, 2, 3, 1, 3, 2, 2, 2, 1], [2, 3, 2, 4, 4, 2, 2, 2, 3, 4, 3, 4, 3, 4, 3, 5, 4, 4, 2, 4, 4, 1, 4, 3, 2, 0, 3, 4, 2, 3, 3, 4, 2, 3, 5, 4, 4, 3, 3, 3, 3, 2, 3, 3, 3, 3, 4, 4, 3], [3, 2, 3, 1, 4, 1, 3, 2, 1, 3, 3, 2, 2, 2, 2, 2, 3, 5, 2, 2, 3, 2, 2, 4, 3, 3, 0, 3, 3, 3, 3, 2, 2, 4, 3, 2, 3, 3, 2, 3, 4, 3, 1, 4, 2, 3, 3, 3, 3], [2, 1, 4, 3, 5, 3, 4, 4, 3, 3, 4, 3, 3, 2, 3, 4, 2, 5, 4, 2, 1, 3, 3, 4, 3, 4, 3, 0, 4, 3, 3, 3, 3, 5, 1, 2, 3, 3, 2, 4, 2, 3, 2, 4, 4, 3, 4, 3, 3], [2, 3, 2, 3, 4, 2, 2, 2, 3, 4, 3, 4, 1, 4, 2, 4, 4, 2, 2, 4, 3, 1, 4, 1, 1, 2, 3, 4, 0, 3, 3, 3, 2, 1, 4, 3, 3, 3, 2, 3, 3, 2, 3, 2, 3, 2, 2, 3, 2], [3, 3, 2, 3, 3, 2, 1, 3, 3, 4, 2, 3, 2, 2, 2, 4, 4, 4, 3, 4, 2, 2, 3, 3, 3, 3, 3, 3, 3, 0, 3, 2, 3, 2, 4, 3, 1, 1, 1, 3, 3, 3, 4, 3, 4, 2, 3, 3, 2], [1, 2, 3, 3, 2, 3, 2, 1, 2, 3, 1, 2, 2, 3, 2, 4, 3, 4, 3, 3, 4, 2, 1, 3, 2, 3, 3, 3, 3, 3, 0, 2, 3, 3, 3, 2, 2, 3, 3, 4, 4, 3, 3, 2, 4, 2, 3, 2, 1], [3, 4, 4, 1, 4, 2, 3, 3, 2, 3, 3, 4, 2, 3, 2, 2, 5, 5, 3, 3, 2, 3, 1, 4, 4, 4, 2, 3, 3, 2, 2, 0, 3, 4, 3, 2, 3, 3, 1, 4, 3, 4, 3, 4, 2, 3, 3, 3, 3], [2, 2, 1, 2, 4, 2, 2, 2, 3, 2, 3, 2, 3, 3, 2, 3, 3, 4, 1, 3, 4, 1, 4, 3, 2, 2, 2, 3, 2, 3, 3, 3, 0, 3, 4, 3, 4, 2, 3, 1, 3, 2, 1, 3, 1, 3, 3, 4, 3], [3, 4, 3, 4, 3, 3, 1, 3, 4, 4, 2, 3, 2, 4, 3, 5, 5, 3, 3, 5, 4, 2, 4, 2, 2, 3, 4, 5, 1, 2, 3, 4, 3, 0, 4, 3, 3, 3, 3, 4, 4, 3, 4, 3, 4, 2, 3, 3, 3], [3, 2, 5, 2, 4, 4, 3, 4, 3, 2, 3, 4, 3, 3, 3, 3, 3, 4, 4, 3, 2, 4, 2, 3, 3, 5, 3, 1, 4, 4, 3, 3, 4, 4, 0, 1, 3, 4, 3, 5, 3, 4, 3, 3, 3, 2, 4, 2, 2], [2, 3, 4, 1, 3, 3, 2, 3, 2, 1, 2, 3, 2, 3, 2, 2, 4, 3, 3, 3, 3, 3, 1, 2, 2, 4, 2, 2, 3, 3, 2, 2, 3, 3, 1, 0, 2, 3, 3, 4, 4, 4, 3, 2, 2, 1, 3, 1, 1], [3, 2, 3, 3, 4, 3, 2, 3, 2, 3, 3, 4, 2, 1, 3, 4, 3, 4, 3, 3, 3, 3, 3, 3, 2, 4, 3, 3, 3, 1, 2, 3, 4, 3, 3, 2, 0, 2, 2, 4, 4, 4, 3, 2, 4, 3, 4, 2, 1], [2, 2, 1, 2, 4, 3, 2, 3, 2, 4, 3, 4, 2, 1, 1, 3, 3, 4, 2, 3, 3, 2, 3, 3, 3, 3, 3, 3, 3, 1, 3, 3, 2, 3, 4, 3, 2, 0, 2, 2, 3, 2, 3, 4, 3, 2, 2, 4, 3], [3, 3, 3, 2, 4, 1, 2, 3, 3, 4, 3, 4, 1, 3, 2, 3, 4, 4, 2, 4, 1, 2, 2, 3, 3, 3, 2, 2, 2, 1, 3, 1, 3, 3, 3, 3, 2, 2, 0, 4, 2, 3, 3, 3, 3, 3, 3, 3, 2], [3, 3, 1, 3, 5, 3, 3, 3, 4, 3, 4, 3, 4, 3, 3, 4, 4, 5, 2, 4, 4, 2, 5, 4, 3, 3, 3, 4, 3, 3, 4, 4, 1, 4, 5, 4, 4, 2, 4, 0, 3, 2, 2, 4, 2, 4, 4, 5, 4], [3, 3, 2, 4, 5, 3, 3, 3, 4, 5, 4, 5, 3, 4, 4, 5, 4, 5, 3, 4, 1, 2, 4, 4, 3, 3, 4, 2, 3, 3, 4, 3, 3, 4, 3, 4, 4, 3, 2, 3, 0, 1, 4, 4, 4, 4, 5, 5, 4], [2, 3, 1, 4, 4, 2, 2, 2, 3, 4, 3, 4, 3, 3, 3, 5, 4, 4, 2, 4, 2, 1, 4, 3, 2, 2, 3, 3, 2, 3, 3, 4, 2, 3, 4, 4, 4, 2, 3, 2, 1, 0, 3, 3, 3, 3, 4, 4, 3], [2, 1, 2, 2, 3, 2, 3, 3, 2, 3, 2, 1, 3, 2, 3, 3, 2, 5, 2, 2, 3, 2, 3, 4, 3, 3, 1, 2, 3, 4, 3, 3, 1, 4, 3, 3, 3, 3, 3, 2, 4, 3, 0, 4, 2, 3, 4, 3, 4], [2, 3, 3, 3, 4, 3, 3, 3, 4, 3, 3, 4, 2, 3, 3, 4, 4, 3, 2, 4, 4, 2, 3, 2, 1, 3, 4, 4, 2, 3, 2, 4, 3, 3, 3, 2, 2, 4, 3, 4, 4, 3, 4, 0, 4, 3, 3, 2, 1], [3, 3, 2, 1, 5, 3, 3, 3, 3, 1, 4, 3, 3, 4, 2, 2, 4, 5, 2, 4, 4, 2, 3, 4, 3, 3, 2, 4, 3, 4, 4, 2, 1, 4, 3, 2, 4, 3, 3, 2, 4, 3, 2, 4, 0, 3, 3, 3, 3], [1, 2, 3, 2, 3, 3, 1, 3, 3, 2, 2, 3, 2, 3, 1, 3, 3, 2, 2, 3, 4, 2, 2, 1, 2, 3, 3, 3, 2, 2, 2, 3, 3, 2, 2, 1, 3, 2, 3, 4, 4, 3, 3, 3, 3, 0, 2, 2, 2], [2, 3, 3, 2, 5, 3, 3, 4, 4, 4, 4, 5, 2, 3, 1, 3, 4, 2, 2, 4, 4, 3, 4, 1, 2, 4, 3, 4, 2, 3, 3, 3, 3, 3, 4, 3, 4, 2, 3, 4, 5, 4, 4, 3, 3, 2, 0, 4, 3], [3, 4, 4, 2, 2, 3, 2, 3, 3, 2, 1, 2, 2, 3, 3, 3, 5, 4, 3, 4, 4, 3, 2, 3, 2, 4, 3, 3, 3, 3, 2, 3, 4, 3, 2, 1, 2, 4, 3, 5, 5, 4, 3, 2, 3, 2, 4, 0, 1], [2, 3, 3, 2, 3, 2, 3, 2, 3, 2, 2, 3, 1, 2, 2, 3, 4, 3, 2, 4, 3, 2, 2, 2, 1, 3, 3, 3, 2, 2, 1, 3, 3, 3, 2, 1, 1, 3, 2, 4, 4, 3, 4, 1, 3, 2, 3, 1, 0]]

    \end{Verbatim}

    \textbf{5. {[}5 marks{]}} Using your function from Question 4, find a
pair of nodes \((i,j)\) in your graph from Part I with the property that
the distance from \(i\) to \(j\) is maximal amongst all pairs of nodes
in the graph.

Note: This means that for every \emph{other} pair of nodes \((i',j')\),
the distance from \(i'\) to \(j'\) is less than or equal to the distance
from \(i\) to \(j\).

    \begin{Verbatim}[commandchars=\\\{\}]
{\color{incolor}In [{\color{incolor}13}]:} \PY{k}{def} \PY{n+nf}{computeMaxDistPair}\PY{p}{(}\PY{n}{dist\PYZus{}matrix}\PY{p}{)}\PY{p}{:}
             \PY{n}{row} \PY{o}{=} \PY{n+nb}{len}\PY{p}{(}\PY{n}{dist\PYZus{}matrix}\PY{p}{)}
             \PY{n}{col} \PY{o}{=} \PY{n+nb}{len}\PY{p}{(}\PY{n}{dist\PYZus{}matrix}\PY{p}{[}\PY{l+m+mi}{0}\PY{p}{]}\PY{p}{)}
             \PY{c+c1}{\PYZsh{} Initialize the pair of nodes(i,j) index as [0,0]}
             \PY{n}{max\PYZus{}pair} \PY{o}{=} \PY{p}{[}\PY{l+m+mi}{0}\PY{p}{]}\PY{o}{*}\PY{l+m+mi}{2}
             \PY{n}{max\PYZus{}val} \PY{o}{=} \PY{n}{dist\PYZus{}matrix}\PY{p}{[}\PY{l+m+mi}{0}\PY{p}{]}\PY{p}{[}\PY{l+m+mi}{0}\PY{p}{]}
             
             \PY{k}{for} \PY{n}{i} \PY{o+ow}{in} \PY{n+nb}{range}\PY{p}{(}\PY{n}{row}\PY{p}{)}\PY{p}{:}
                 \PY{k}{for} \PY{n}{j} \PY{o+ow}{in} \PY{n+nb}{range}\PY{p}{(}\PY{n}{i}\PY{p}{,} \PY{n}{col}\PY{p}{)}\PY{p}{:}
                     \PY{k}{if} \PY{n}{dist\PYZus{}matrix}\PY{p}{[}\PY{n}{i}\PY{p}{]}\PY{p}{[}\PY{n}{j}\PY{p}{]} \PY{o}{\PYZgt{}} \PY{n}{max\PYZus{}val}\PY{p}{:}
                         \PY{n}{max\PYZus{}val} \PY{o}{=} \PY{n}{dist\PYZus{}matrix}\PY{p}{[}\PY{n}{i}\PY{p}{]}\PY{p}{[}\PY{n}{j}\PY{p}{]}
                         \PY{n}{max\PYZus{}pair} \PY{o}{=} \PY{p}{[}\PY{n}{i}\PY{p}{,} \PY{n}{j}\PY{p}{]}
             \PY{k}{return} \PY{n}{max\PYZus{}val}\PY{p}{,} \PY{n}{max\PYZus{}pair}
                     
\end{Verbatim}


    \begin{Verbatim}[commandchars=\\\{\}]
{\color{incolor}In [{\color{incolor}14}]:} \PY{n}{max\PYZus{}val} \PY{o}{=} \PY{n}{computeMaxDistPair}\PY{p}{(}\PY{n}{dist\PYZus{}matrix}\PY{p}{)}\PY{p}{[}\PY{l+m+mi}{0}\PY{p}{]}
         \PY{n}{max\PYZus{}pair} \PY{o}{=} \PY{n}{computeMaxDistPair}\PY{p}{(}\PY{n}{dist\PYZus{}matrix}\PY{p}{)}\PY{p}{[}\PY{l+m+mi}{1}\PY{p}{]}
         \PY{n+nb}{print}\PY{p}{(}\PY{l+s+s2}{\PYZdq{}}\PY{l+s+s2}{The maximal value amongst all pairs of nodes is}\PY{l+s+s2}{\PYZdq{}}\PY{p}{,} \PY{n}{max\PYZus{}val}\PY{p}{,} \PY{l+s+s2}{\PYZdq{}}\PY{l+s+s2}{,}\PY{l+s+se}{\PYZbs{}n}\PY{l+s+s2}{the index of nodes is : }\PY{l+s+s2}{\PYZdq{}}\PY{p}{,} \PY{n}{max\PYZus{}pair}\PY{p}{)}
\end{Verbatim}


    \begin{Verbatim}[commandchars=\\\{\}]
The maximal value amongst all pairs of nodes is 5 ,
the index of nodes is :  [2, 34]

    \end{Verbatim}

    \textbf{6. {[}5 marks{]}} Find a shortest path between your nodes from
Question 5, i.e. one with the shortest possible length, and re-draw your
graph so that this path is clearly visible. You should use one colour
for the nodes and edges in the path, and a different colour for all
other nodes and edges.

Hint: You should be able to find a NetworkX function that computes a
shortest path.

    \begin{Verbatim}[commandchars=\\\{\}]
{\color{incolor}In [{\color{incolor}15}]:} \PY{k}{def} \PY{n+nf}{printShortestPath}\PY{p}{(}\PY{n}{G}\PY{p}{)}\PY{p}{:}
             \PY{n}{shortest\PYZus{}paths} \PY{o}{=}\PY{n}{nx}\PY{o}{.}\PY{n}{shortest\PYZus{}path}\PY{p}{(}\PY{n}{G}\PY{p}{,} \PY{n}{source}\PY{o}{=}\PY{l+m+mi}{2}\PY{p}{,} \PY{n}{target}\PY{o}{=}\PY{l+m+mi}{34}\PY{p}{)}
             \PY{k}{return} \PY{n}{shortest\PYZus{}paths}
\end{Verbatim}


    \begin{Verbatim}[commandchars=\\\{\}]
{\color{incolor}In [{\color{incolor}196}]:} \PY{n}{shortest\PYZus{}paths} \PY{o}{=} \PY{n}{printShortestPath}\PY{p}{(}\PY{n}{G}\PY{p}{)}
          \PY{n+nb}{print}\PY{p}{(}\PY{n}{shortest\PYZus{}paths}\PY{p}{)}
          
          \PY{n}{pos} \PY{o}{=} \PY{n}{nx}\PY{o}{.}\PY{n}{spring\PYZus{}layout}\PY{p}{(}\PY{n}{G}\PY{p}{)}
          \PY{n}{plt}\PY{o}{.}\PY{n}{figure}\PY{p}{(}\PY{n}{figsize}\PY{o}{=}\PY{p}{(}\PY{l+m+mi}{18}\PY{p}{,}\PY{l+m+mi}{18}\PY{p}{)}\PY{p}{)}
          \PY{c+c1}{\PYZsh{} nx.draw(G,pos,node\PYZus{}color=\PYZsq{}k\PYZsq{})}
          \PY{n}{nx}\PY{o}{.}\PY{n}{draw\PYZus{}networkx}\PY{p}{(}\PY{n}{G}\PY{p}{,} \PY{n}{pos}\PY{p}{,} \PY{n}{node\PYZus{}color} \PY{o}{=} \PY{l+s+s1}{\PYZsq{}}\PY{l+s+s1}{Coral}\PY{l+s+s1}{\PYZsq{}}\PY{p}{,} \PY{n}{edge\PYZus{}color} \PY{o}{=} \PY{l+s+s1}{\PYZsq{}}\PY{l+s+s1}{black}\PY{l+s+s1}{\PYZsq{}}\PY{p}{,} \PY{n}{with\PYZus{}labels}\PY{o}{=}\PY{k+kc}{True}\PY{p}{)}
          \PY{n}{path} \PY{o}{=} \PY{n}{nx}\PY{o}{.}\PY{n}{shortest\PYZus{}path}\PY{p}{(}\PY{n}{G}\PY{p}{,}\PY{n}{source}\PY{o}{=}\PY{l+m+mi}{2}\PY{p}{,}\PY{n}{target}\PY{o}{=}\PY{l+m+mi}{34}\PY{p}{)}
          \PY{n}{path\PYZus{}edges} \PY{o}{=} \PY{n+nb}{zip}\PY{p}{(}\PY{n}{path}\PY{p}{,}\PY{n}{path}\PY{p}{[}\PY{l+m+mi}{1}\PY{p}{:}\PY{p}{]}\PY{p}{)}
          \PY{n}{path\PYZus{}edges} \PY{o}{=} \PY{n+nb}{set}\PY{p}{(}\PY{n}{path\PYZus{}edges}\PY{p}{)}
          \PY{n}{nx}\PY{o}{.}\PY{n}{draw\PYZus{}networkx\PYZus{}nodes}\PY{p}{(}\PY{n}{G}\PY{p}{,}\PY{n}{pos}\PY{p}{,}\PY{n}{nodelist}\PY{o}{=}\PY{n}{path}\PY{p}{,}\PY{n}{node\PYZus{}color}\PY{o}{=}\PY{l+s+s1}{\PYZsq{}}\PY{l+s+s1}{Coral}\PY{l+s+s1}{\PYZsq{}}\PY{p}{)}
          \PY{n}{nx}\PY{o}{.}\PY{n}{draw\PYZus{}networkx\PYZus{}edges}\PY{p}{(}\PY{n}{G}\PY{p}{,}\PY{n}{pos}\PY{p}{,}\PY{n}{edgelist}\PY{o}{=}\PY{n}{path\PYZus{}edges}\PY{p}{,}\PY{n}{edge\PYZus{}color}\PY{o}{=}\PY{l+s+s1}{\PYZsq{}}\PY{l+s+s1}{RoyalBlue}\PY{l+s+s1}{\PYZsq{}}\PY{p}{,}\PY{n}{width}\PY{o}{=}\PY{l+m+mi}{5}\PY{p}{)}
          \PY{n}{plt}\PY{o}{.}\PY{n}{xticks}\PY{p}{(}\PY{p}{[}\PY{p}{]}\PY{p}{)}
          \PY{n}{plt}\PY{o}{.}\PY{n}{yticks}\PY{p}{(}\PY{p}{[}\PY{p}{]}\PY{p}{)}
          \PY{n}{plt}\PY{o}{.}\PY{n}{show}\PY{p}{(}\PY{p}{)}
\end{Verbatim}


    \begin{Verbatim}[commandchars=\\\{\}]
[2, 32, 44, 9, 35, 34]

    \end{Verbatim}

    \begin{center}
    \adjustimage{max size={0.9\linewidth}{0.9\paperheight}}{output_46_1.png}
    \end{center}
    { \hspace*{\fill} \\}
    
    \subsubsection{Part III: Laplacian matrices and spanning trees {[}30
marks{]}}\label{part-iii-laplacian-matrices-and-spanning-trees-30-marks}

So far you have learned about two matrices associated with a graph: the
adjacency matrix, and the distance matrix. Now you will study a third
matrix: the Laplacian.

\textbf{7. {[}10 marks{]}} Explain what the "degree" of a node in a
graph is, what the "Laplacian matrix" of a graph is, what a "spanning
tree" of a graph is, and how the Laplacian matrix can be used to
calculate the number of spanning trees in a graph. Here, again, you
should discuss arbitrary (undirected, simple, connected) graphs, not
your specific graph from Part I.

Note: You do \textbf{not} need to give any proofs, but you must
reference any material you use, as explained in the plagiarism warning
above.

    \textbf{Answer}: - the "degree" of a node in a graph: - the degree of a
node in a graph is the number of edges connected to this node. Usually,
we count the number of edges which has the node as an endpoint. In
addition, the degree of a graph is the largest node degree among all the
nodes in the graph.

\begin{itemize}
\tightlist
\item
  the "Laplacian matrix" of a graph:

  \begin{itemize}
  \tightlist
  \item
    For a simple graph G with n nodes, the Laplacian matrix of this
    graph is L = D - A where D is the degree matrix and A is the
    adjacency matrix of this graph.
  \item
    For undirected, unweighted graph with n nodes, without loop from one
    node to itself or multiple edges from one node to another node, the
    Laplacian matrix of this graph is L = D - A.
  \item
    Briefly, the "Laplacian matrix" of a graph is equal to the
    difference between the degree matrix of this graph and its adjacency
    matrix.
  \end{itemize}
\item
  a "spanning tree" of a graph:

  \begin{itemize}
  \tightlist
  \item
    a "spanning tree" of a graph G could be considered as a subset of G,
    that is a tree, which contains all the nodes of G, where all of the
    nodes are covered with minimum number of edges. a spanning tree does
    not have loops and it cannot be disconnected. A graph might have
    several spanning trees (the graph should not be connected in order
    to have a spanning tree.)
  \end{itemize}
\item
  how the Laplacian matrix can be used to calculate the number of
  spanning trees in a graph

  \begin{itemize}
  \tightlist
  \item
    According to Kirchhoff's theorem which relates the number of
    spanning trees of a graph with its Laplacian matrix, the number of
    spanning trees of a graph G is equal to any cofactor of the
    Laplacian matrix of G. Represents as:
  \end{itemize}
\end{itemize}

    \begin{Verbatim}[commandchars=\\\{\}]
{\color{incolor}In [{\color{incolor}19}]:} \PY{k+kn}{from} \PY{n+nn}{IPython}\PY{n+nn}{.}\PY{n+nn}{display} \PY{k}{import} \PY{n}{Latex}
         \PY{n}{Latex}\PY{p}{(}\PY{l+s+sa}{r}\PY{l+s+s2}{\PYZdq{}}\PY{l+s+s2}{\PYZdl{}t(G) = }\PY{l+s+s2}{\PYZbs{}}\PY{l+s+s2}{frac}\PY{l+s+si}{\PYZob{}1\PYZcb{}}\PY{l+s+si}{\PYZob{}n\PYZcb{}}\PY{l+s+s2}{* }\PY{l+s+s2}{\PYZbs{}}\PY{l+s+s2}{lambda\PYZus{}1 }\PY{l+s+s2}{\PYZbs{}}\PY{l+s+s2}{lambda\PYZus{}2 ... }\PY{l+s+s2}{\PYZbs{}}\PY{l+s+s2}{lambda\PYZus{}}\PY{l+s+s2}{\PYZob{}}\PY{l+s+s2}{n\PYZhy{}1\PYZcb{}\PYZdl{}}\PY{l+s+s2}{\PYZdq{}}\PY{p}{)}
\end{Verbatim}

\texttt{\color{outcolor}Out[{\color{outcolor}19}]:}
    
    $t(G) = \frac{1}{n}* \lambda_1 \lambda_2 ... \lambda_{n-1}$

    

    where \(\lambda_1\), \(\lambda_2\) ,..., \(\lambda_{n-1}\) are non-zero
eigenvalues of the Laplacian matrix.

    \textbf{8. {[}10 marks{]}} Write a function
\texttt{number\_of\_spanning\_trees} which takes as input a graph \(G\)
and returns the number of spanning trees in \(G\). You may use the
NetworkX function \texttt{adjacency\_matrix} to compute the adjacency
matrix of the input graph, but you \textbf{may not use any other
NetworkX functions}.

Note: You will probably need to compute the determinant of a certain
matrix somewhere in your code. If you use the function
\texttt{numpy.linalg.det} then your determinant will only be computed
approximately, i.e. to a certain numerical precision. This is fine; you
will not lose any marks if your code is otherwise correct.

    \begin{Verbatim}[commandchars=\\\{\}]
{\color{incolor}In [{\color{incolor}33}]:} \PY{k}{def} \PY{n+nf}{find\PYZus{}degree}\PY{p}{(}\PY{n}{G}\PY{p}{,} \PY{n}{node}\PY{p}{)}\PY{p}{:}
             \PY{n}{adj\PYZus{}matrix} \PY{o}{=} \PY{n}{nx}\PY{o}{.}\PY{n}{adjacency\PYZus{}matrix}\PY{p}{(}\PY{n}{G}\PY{p}{)}\PY{o}{.}\PY{n}{toarray}\PY{p}{(}\PY{p}{)}
             \PY{n}{num\PYZus{}nodes} \PY{o}{=} \PY{n}{G}\PY{o}{.}\PY{n}{number\PYZus{}of\PYZus{}nodes}\PY{p}{(}\PY{p}{)}
             \PY{n}{degree} \PY{o}{=} \PY{l+m+mi}{0}
             \PY{k}{for} \PY{n}{d} \PY{o+ow}{in} \PY{n+nb}{range}\PY{p}{(}\PY{n}{num\PYZus{}nodes}\PY{p}{)}\PY{p}{:}
                 \PY{k}{if} \PY{n}{adj\PYZus{}matrix}\PY{p}{[}\PY{n}{node}\PY{p}{]}\PY{p}{[}\PY{n}{d}\PY{p}{]} \PY{o}{==} \PY{l+m+mi}{1}\PY{p}{:}
                     \PY{n}{degree} \PY{o}{=} \PY{n}{degree} \PY{o}{+} \PY{l+m+mi}{1}
             \PY{k}{return} \PY{n}{degree}
\end{Verbatim}


    \begin{Verbatim}[commandchars=\\\{\}]
{\color{incolor}In [{\color{incolor}45}]:} \PY{c+c1}{\PYZsh{} Cayley’s formula}
         \PY{k+kn}{import} \PY{n+nn}{math}
         \PY{k}{def} \PY{n+nf}{number\PYZus{}of\PYZus{}spanning\PYZus{}trees}\PY{p}{(}\PY{n}{G}\PY{p}{)}\PY{p}{:}
             \PY{n}{adj\PYZus{}matrix} \PY{o}{=} \PY{n}{nx}\PY{o}{.}\PY{n}{adjacency\PYZus{}matrix}\PY{p}{(}\PY{n}{G}\PY{p}{)}\PY{o}{.}\PY{n}{toarray}\PY{p}{(}\PY{p}{)}
             \PY{n}{num\PYZus{}nodes} \PY{o}{=} \PY{n}{G}\PY{o}{.}\PY{n}{number\PYZus{}of\PYZus{}nodes}\PY{p}{(}\PY{p}{)}
             \PY{c+c1}{\PYZsh{} Replace all the diagonal elements with the degree of nodes}
             \PY{k}{for} \PY{n}{n} \PY{o+ow}{in} \PY{n+nb}{range}\PY{p}{(}\PY{n}{num\PYZus{}nodes}\PY{p}{)}\PY{p}{:}
                 \PY{n}{adj\PYZus{}matrix}\PY{p}{[}\PY{n}{n}\PY{p}{]}\PY{p}{[}\PY{n}{n}\PY{p}{]} \PY{o}{=} \PY{n}{find\PYZus{}degree}\PY{p}{(}\PY{n}{G}\PY{p}{,} \PY{n}{n}\PY{p}{)}
                 
             \PY{c+c1}{\PYZsh{} Replace all non\PYZhy{}diagonal 1’s with \PYZhy{}1.}
             \PY{k}{for} \PY{n}{r} \PY{o+ow}{in} \PY{n+nb}{range}\PY{p}{(}\PY{n}{num\PYZus{}nodes}\PY{p}{)}\PY{p}{:}
                 \PY{k}{for} \PY{n}{c} \PY{o+ow}{in} \PY{n+nb}{range}\PY{p}{(}\PY{n}{num\PYZus{}nodes}\PY{p}{)}\PY{p}{:}
                     \PY{k}{if} \PY{n}{adj\PYZus{}matrix}\PY{p}{[}\PY{n}{r}\PY{p}{]}\PY{p}{[}\PY{n}{c}\PY{p}{]} \PY{o}{==} \PY{l+m+mi}{1}\PY{p}{:}
                         \PY{n}{adj\PYZus{}matrix}\PY{p}{[}\PY{n}{r}\PY{p}{]}\PY{p}{[}\PY{n}{c}\PY{p}{]} \PY{o}{=} \PY{o}{\PYZhy{}}\PY{l+m+mi}{1}
                 
             \PY{n+nb}{print}\PY{p}{(}\PY{l+s+s2}{\PYZdq{}}\PY{l+s+s2}{Matrix: }\PY{l+s+s2}{\PYZdq{}}\PY{p}{,} \PY{n}{adj\PYZus{}matrix}\PY{p}{)}
             \PY{c+c1}{\PYZsh{} Calculate co\PYZhy{}factor for any element.}
             \PY{n}{det} \PY{o}{=} \PY{n}{np}\PY{o}{.}\PY{n}{linalg}\PY{o}{.}\PY{n}{det}\PY{p}{(}\PY{n}{adj\PYZus{}matrix}\PY{p}{)}
             \PY{n+nb}{print}\PY{p}{(}\PY{l+s+s2}{\PYZdq{}}\PY{l+s+s2}{co\PYZhy{}factor}\PY{l+s+s2}{\PYZdq{}}\PY{p}{,} \PY{n}{math}\PY{o}{.}\PY{n}{pow}\PY{p}{(}\PY{o}{\PYZhy{}}\PY{l+m+mi}{1}\PY{p}{,} \PY{l+m+mi}{1}\PY{p}{)}\PY{o}{*}\PY{n}{det}\PY{p}{)}
             \PY{k}{return} \PY{n}{math}\PY{o}{.}\PY{n}{pow}\PY{p}{(}\PY{o}{\PYZhy{}}\PY{l+m+mi}{1}\PY{p}{,} \PY{l+m+mi}{1}\PY{p}{)}\PY{o}{*}\PY{n}{det}
             
             
\end{Verbatim}


    \textbf{9 {[}5 marks{]}} Use your function from Question 8 to calculate
the number of spanning trees in your graph from Part I.

    \begin{Verbatim}[commandchars=\\\{\}]
{\color{incolor}In [{\color{incolor}46}]:} \PY{n}{number\PYZus{}of\PYZus{}spanning\PYZus{}trees}\PY{p}{(}\PY{n}{G}\PY{p}{)}
\end{Verbatim}


    \begin{Verbatim}[commandchars=\\\{\}]
Matrix:  [[ 6 -1  0 {\ldots}  0  0  0]
 [-1  6  0 {\ldots}  0  0  0]
 [ 0  0  5 {\ldots}  0  0  0]
 {\ldots}
 [ 0  0  0 {\ldots}  2  0  0]
 [ 0  0  0 {\ldots}  0  3 -1]
 [ 0  0  0 {\ldots}  0 -1  7]]
co-factor 1.1961427653389379e+23

    \end{Verbatim}

\begin{Verbatim}[commandchars=\\\{\}]
{\color{outcolor}Out[{\color{outcolor}46}]:} 1.1961427653389379e+23
\end{Verbatim}
            
    \textbf{10 {[}5 marks{]}} Find a minimal spanning tree of your graph
from Part I, i.e. one with the smallest possible number of edges.
Re-draw your graph in such a way that this spanning tree is clearly
visible. You should use one colour for the edges in the spanning tree,
and a different colour for all other edges.

Hint: You should be able to find a NetworkX function that computes a
minimal spanning tree.

    \begin{Verbatim}[commandchars=\\\{\}]
{\color{incolor}In [{\color{incolor}26}]:} \PY{k}{def} \PY{n+nf}{primsAlg}\PY{p}{(}\PY{n}{G}\PY{p}{)}\PY{p}{:}
             \PY{n}{adj\PYZus{}matrix} \PY{o}{=} \PY{n}{nx}\PY{o}{.}\PY{n}{adjacency\PYZus{}matrix}\PY{p}{(}\PY{n}{G}\PY{p}{)}\PY{o}{.}\PY{n}{toarray}\PY{p}{(}\PY{p}{)}
             \PY{n}{num\PYZus{}nodes} \PY{o}{=} \PY{n}{G}\PY{o}{.}\PY{n}{number\PYZus{}of\PYZus{}nodes}\PY{p}{(}\PY{p}{)}
             \PY{c+c1}{\PYZsh{} arbitrarily choose initial node from graph}
             \PY{n}{node} \PY{o}{=} \PY{l+m+mi}{0}
             \PY{c+c1}{\PYZsh{} initialize empty edges array and empty MST}
             \PY{n}{MST} \PY{o}{=} \PY{p}{[}\PY{p}{]}\PY{p}{;}
             \PY{n}{edges} \PY{o}{=} \PY{p}{[}\PY{p}{]}\PY{p}{;}
             \PY{n}{visited} \PY{o}{=} \PY{p}{[}\PY{p}{]}\PY{p}{;}
             \PY{n}{miniEdge} \PY{o}{=} \PY{p}{[}\PY{k+kc}{None}\PY{p}{,}\PY{k+kc}{None}\PY{p}{,}\PY{n+nb}{float}\PY{p}{(}\PY{l+s+s2}{\PYZdq{}}\PY{l+s+s2}{Inf}\PY{l+s+s2}{\PYZdq{}}\PY{p}{)}\PY{p}{]}\PY{p}{;}
             
             \PY{c+c1}{\PYZsh{} run prims algorithm }
             \PY{c+c1}{\PYZsh{} until we create an MST that contains every vertex from the graph}
             \PY{k}{while} \PY{p}{(}\PY{n+nb}{len}\PY{p}{(}\PY{n}{MST}\PY{p}{)} \PY{o}{!=} \PY{n}{num\PYZus{}nodes}\PY{o}{\PYZhy{}}\PY{l+m+mi}{1}\PY{p}{)}\PY{p}{:}
                 \PY{c+c1}{\PYZsh{} mark this vertex as visited}
                 \PY{n}{visited}\PY{o}{.}\PY{n}{append}\PY{p}{(}\PY{n}{node}\PY{p}{)}
                 
                 \PY{k}{for} \PY{n}{n} \PY{o+ow}{in} \PY{n+nb}{range}\PY{p}{(}\PY{l+m+mi}{0}\PY{p}{,} \PY{n}{num\PYZus{}nodes}\PY{p}{)}\PY{p}{:}
                     \PY{k}{if} \PY{n}{adj\PYZus{}matrix}\PY{p}{[}\PY{n}{node}\PY{p}{]}\PY{p}{[}\PY{n}{n}\PY{p}{]} \PY{o}{!=} \PY{l+m+mi}{0}\PY{p}{:}
                         \PY{n}{edges}\PY{o}{.}\PY{n}{append}\PY{p}{(}\PY{p}{[}\PY{n}{node}\PY{p}{,}\PY{n}{n}\PY{p}{,}\PY{n}{adj\PYZus{}matrix}\PY{p}{[}\PY{n}{node}\PY{p}{]}\PY{p}{[}\PY{n}{n}\PY{p}{]}\PY{p}{]}\PY{p}{)}
                 
                 \PY{c+c1}{\PYZsh{} find the edge that has the smallest weight to an un\PYZhy{}visited node}
                 \PY{k}{for} \PY{n}{e} \PY{o+ow}{in} \PY{n+nb}{range}\PY{p}{(}\PY{l+m+mi}{0}\PY{p}{,} \PY{n+nb}{len}\PY{p}{(}\PY{n}{edges}\PY{p}{)}\PY{p}{)}\PY{p}{:}
                     \PY{k}{if} \PY{n}{edges}\PY{p}{[}\PY{n}{e}\PY{p}{]}\PY{p}{[}\PY{l+m+mi}{2}\PY{p}{]} \PY{o}{\PYZlt{}} \PY{n}{miniEdge}\PY{p}{[}\PY{l+m+mi}{2}\PY{p}{]} \PY{o+ow}{and} \PY{n}{edges}\PY{p}{[}\PY{n}{e}\PY{p}{]}\PY{p}{[}\PY{l+m+mi}{1}\PY{p}{]} \PY{o+ow}{not} \PY{o+ow}{in} \PY{n}{visited}\PY{p}{:}
                         \PY{n}{miniEdge} \PY{o}{=} \PY{n}{edges}\PY{p}{[}\PY{n}{e}\PY{p}{]}
                         
                 \PY{n}{edges}\PY{o}{.}\PY{n}{remove}\PY{p}{(}\PY{n}{miniEdge}\PY{p}{)}
                 \PY{n}{MST}\PY{o}{.}\PY{n}{append}\PY{p}{(}\PY{n}{miniEdge}\PY{p}{)}
                 
                 \PY{c+c1}{\PYZsh{} restart, at a new node and reset the mini edge}
                 \PY{n}{node} \PY{o}{=} \PY{n}{miniEdge}\PY{p}{[}\PY{l+m+mi}{1}\PY{p}{]}
                 \PY{n}{miniEdge} \PY{o}{=} \PY{p}{[}\PY{k+kc}{None}\PY{p}{,}\PY{k+kc}{None}\PY{p}{,}\PY{n+nb}{float}\PY{p}{(}\PY{l+s+s2}{\PYZdq{}}\PY{l+s+s2}{Inf}\PY{l+s+s2}{\PYZdq{}}\PY{p}{)}\PY{p}{]}\PY{p}{;}
             \PY{k}{return} \PY{n}{MST}
\end{Verbatim}


    \begin{Verbatim}[commandchars=\\\{\}]
{\color{incolor}In [{\color{incolor}25}]:} \PY{n}{primsAlg}\PY{p}{(}\PY{n}{G}\PY{p}{)}
\end{Verbatim}


\begin{Verbatim}[commandchars=\\\{\}]
{\color{outcolor}Out[{\color{outcolor}25}]:} [[0, 1, 1],
          [0, 14, 1],
          [0, 21, 1],
          [0, 24, 1],
          [0, 30, 1],
          [0, 45, 1],
          [1, 13, 1],
          [1, 16, 1],
          [1, 19, 1],
          [1, 27, 1],
          [1, 42, 1],
          [14, 3, 1],
          [14, 12, 1],
          [14, 18, 1],
          [14, 37, 1],
          [14, 46, 1],
          [21, 2, 1],
          [21, 5, 1],
          [21, 6, 1],
          [21, 7, 1],
          [21, 25, 1],
          [21, 28, 1],
          [21, 32, 1],
          [21, 41, 1],
          [24, 23, 1],
          [24, 43, 1],
          [24, 48, 1],
          [30, 10, 1],
          [30, 22, 1],
          [45, 35, 1],
          [13, 8, 1],
          [13, 36, 1],
          [27, 20, 1],
          [27, 34, 1],
          [42, 11, 1],
          [42, 26, 1],
          [3, 15, 1],
          [3, 31, 1],
          [3, 44, 1],
          [12, 38, 1],
          [37, 29, 1],
          [2, 39, 1],
          [6, 33, 1],
          [41, 40, 1],
          [23, 17, 1],
          [48, 47, 1],
          [10, 4, 1],
          [35, 9, 1]]
\end{Verbatim}
            
    \begin{Verbatim}[commandchars=\\\{\}]
{\color{incolor}In [{\color{incolor}189}]:} \PY{c+c1}{\PYZsh{} Kruskal’s algorithm}
          \PY{n}{mst} \PY{o}{=} \PY{n}{nx}\PY{o}{.}\PY{n}{minimum\PYZus{}spanning\PYZus{}edges}\PY{p}{(}\PY{n}{G}\PY{p}{,}\PY{n}{data}\PY{o}{=}\PY{k+kc}{False}\PY{p}{)} \PY{c+c1}{\PYZsh{} a generator of MST edges}
          \PY{n}{edgelist}\PY{o}{=}\PY{n+nb}{list}\PY{p}{(}\PY{n}{mst}\PY{p}{)} \PY{c+c1}{\PYZsh{} make a list of the edges}
          \PY{n+nb}{print}\PY{p}{(}\PY{n+nb}{sorted}\PY{p}{(}\PY{n}{edgelist}\PY{p}{)}\PY{p}{)}
          \PY{n}{path\PYZus{}edges} \PY{o}{=} \PY{n+nb}{set}\PY{p}{(}\PY{n}{edgelist}\PY{p}{)}
          \PY{n}{pos} \PY{o}{=} \PY{n}{nx}\PY{o}{.}\PY{n}{spring\PYZus{}layout}\PY{p}{(}\PY{n}{G}\PY{p}{)}
          \PY{n}{plt}\PY{o}{.}\PY{n}{figure}\PY{p}{(}\PY{n}{figsize}\PY{o}{=}\PY{p}{(}\PY{l+m+mi}{18}\PY{p}{,}\PY{l+m+mi}{18}\PY{p}{)}\PY{p}{)}
          \PY{n}{nx}\PY{o}{.}\PY{n}{draw\PYZus{}networkx}\PY{p}{(}\PY{n}{G}\PY{p}{,} \PY{n}{pos}\PY{p}{,} \PY{n}{node\PYZus{}color} \PY{o}{=} \PY{l+s+s1}{\PYZsq{}}\PY{l+s+s1}{Coral}\PY{l+s+s1}{\PYZsq{}}\PY{p}{,} \PY{n}{edge\PYZus{}color} \PY{o}{=} \PY{l+s+s1}{\PYZsq{}}\PY{l+s+s1}{black}\PY{l+s+s1}{\PYZsq{}}\PY{p}{,} \PY{n}{with\PYZus{}labels}\PY{o}{=}\PY{k+kc}{True}\PY{p}{)}
          \PY{n}{nx}\PY{o}{.}\PY{n}{draw\PYZus{}networkx\PYZus{}edges}\PY{p}{(}\PY{n}{G}\PY{p}{,}\PY{n}{pos}\PY{p}{,}\PY{n}{edgelist}\PY{o}{=}\PY{n}{path\PYZus{}edges}\PY{p}{,}\PY{n}{edge\PYZus{}color}\PY{o}{=}\PY{l+s+s1}{\PYZsq{}}\PY{l+s+s1}{RoyalBlue}\PY{l+s+s1}{\PYZsq{}}\PY{p}{,}\PY{n}{width}\PY{o}{=}\PY{l+m+mi}{5}\PY{p}{)}
\end{Verbatim}


    \begin{Verbatim}[commandchars=\\\{\}]
[(0, 1), (0, 14), (0, 21), (0, 24), (0, 30), (0, 45), (1, 13), (1, 16), (1, 19), (1, 27), (1, 42), (2, 21), (2, 32), (2, 37), (2, 39), (2, 41), (3, 14), (3, 15), (3, 26), (3, 31), (3, 35), (3, 44), (4, 10), (5, 12), (5, 18), (5, 26), (5, 38), (6, 10), (6, 29), (6, 33), (6, 45), (7, 8), (7, 21), (8, 22), (9, 35), (10, 11), (10, 47), (12, 28), (12, 48), (13, 36), (14, 46), (17, 23), (20, 27), (20, 40), (21, 25), (23, 24), (24, 43), (27, 34)]

    \end{Verbatim}

\begin{Verbatim}[commandchars=\\\{\}]
{\color{outcolor}Out[{\color{outcolor}189}]:} <matplotlib.collections.LineCollection at 0x10d42ad30>
\end{Verbatim}
            
    \begin{center}
    \adjustimage{max size={0.9\linewidth}{0.9\paperheight}}{output_59_2.png}
    \end{center}
    { \hspace*{\fill} \\}
    
    \subsubsection{Part IV: Eigenvalues and triangles {[}30
marks{]}}\label{part-iv-eigenvalues-and-triangles-30-marks}

By now you have seen that certain matrices associated with a graph can
tell us a lot about the structure of the graph. In this final part of
the project, you will investigate this further, by learning how
eigenvalues can be used to reveal even more information about graphs.

\textbf{11. {[}5 marks{]}} Explain what a "triangle" in a graph is, and
quote a formula for calculating the number of triangles in a graph from
the eigenvalues of the adjacency matrix.

Note: You do \textbf{not} need to give any proofs, but you must
reference any material you use, as explained in the plagiarism warning
above.

    \textbf{Answer}: - a "triangle" in a graph: - In an undirected graph, a
triangle of a graph is formed by 3 nodes adjacent in G that connected by
3 edges, which makes itself a cycle graph, a complete graph.

\begin{itemize}
\item
  Formula to calculate the number of triangles:

  \begin{itemize}
  \tightlist
  \item
    Algorithm: each node, v, of the graph G:

    \begin{itemize}
    \tightlist
    \item
      for each pair u, w in nodes(v), where u and w are distinct
      neighbors of v that both have higher degree than v:

      \begin{itemize}
      \tightlist
      \item
        if u,v,w form a triangle, increment a running count of the
        triangles of the graph.
      \end{itemize}
    \end{itemize}
  \item
    references:

    \begin{itemize}
    \tightlist
    \item
      {[}1{]} Suri, S. and Vassilvitskii, S., 2011, March. Counting
      triangles and the curse of the last reducer. In Proceedings of the
      20th international conference on World wide web (pp. 607-614).
      ACM.
    \item
      {[}2{]} Kolountzakis, M.N., Miller, G.L., Peng, R. and
      Tsourakakis, C.E., 2012. Efficient triangle counting in large
      graphs via degree-based vertex partitioning. Internet Mathematics,
      8(1-2), pp.161-185.
    \end{itemize}
  \end{itemize}
\end{itemize}

    \textbf{12. {[}5 marks{]}} Calculate the number of triangles in your
graph from Part I, using the formula discussed in question 11. Your
answer \textbf{must} be an integer.

Hint: What is the "trace" of a matrix and how is it related to the
eigenvalues?

    \begin{Verbatim}[commandchars=\\\{\}]
{\color{incolor}In [{\color{incolor}106}]:} \PY{c+c1}{\PYZsh{} matrix multiplecation}
          \PY{k}{def} \PY{n+nf}{matrix\PYZus{}multiple}\PY{p}{(}\PY{n}{G}\PY{p}{,} \PY{n}{M1}\PY{p}{,} \PY{n}{M2}\PY{p}{,} \PY{n}{M3}\PY{p}{)}\PY{p}{:}
              \PY{n}{num\PYZus{}nodes} \PY{o}{=} \PY{n}{G}\PY{o}{.}\PY{n}{number\PYZus{}of\PYZus{}nodes}\PY{p}{(}\PY{p}{)}
              
              \PY{k}{for} \PY{n}{r} \PY{o+ow}{in} \PY{n+nb}{range}\PY{p}{(}\PY{n}{num\PYZus{}nodes}\PY{p}{)}\PY{p}{:}
                  \PY{k}{for} \PY{n}{c} \PY{o+ow}{in} \PY{n+nb}{range}\PY{p}{(}\PY{n}{num\PYZus{}nodes}\PY{p}{)}\PY{p}{:}
                      \PY{n}{M3}\PY{p}{[}\PY{n}{r}\PY{p}{]}\PY{p}{[}\PY{n}{c}\PY{p}{]} \PY{o}{=} \PY{l+m+mi}{0}
                      \PY{k}{for} \PY{n}{i} \PY{o+ow}{in} \PY{n+nb}{range}\PY{p}{(}\PY{n}{num\PYZus{}nodes}\PY{p}{)}\PY{p}{:}
                          \PY{n}{M3}\PY{p}{[}\PY{n}{r}\PY{p}{]}\PY{p}{[}\PY{n}{c}\PY{p}{]} \PY{o}{+}\PY{o}{=} \PY{n}{M1}\PY{p}{[}\PY{n}{r}\PY{p}{]}\PY{p}{[}\PY{n}{i}\PY{p}{]} \PY{o}{*} \PY{n}{M2}\PY{p}{[}\PY{n}{i}\PY{p}{]}\PY{p}{[}\PY{n}{c}\PY{p}{]}
              \PY{k}{return} \PY{n}{M3}
\end{Verbatim}


    \begin{Verbatim}[commandchars=\\\{\}]
{\color{incolor}In [{\color{incolor}107}]:} \PY{c+c1}{\PYZsh{} calculate the trace of a matrix (sum of diagnonal elements)}
          \PY{k}{def} \PY{n+nf}{find\PYZus{}trace}\PY{p}{(}\PY{n}{G}\PY{p}{,}\PY{n}{adj\PYZus{}matrix}\PY{p}{)}\PY{p}{:}
              \PY{n}{trace} \PY{o}{=} \PY{l+m+mi}{0}
              \PY{n}{num\PYZus{}nodes} \PY{o}{=} \PY{n}{G}\PY{o}{.}\PY{n}{number\PYZus{}of\PYZus{}nodes}\PY{p}{(}\PY{p}{)}
              \PY{k}{for} \PY{n}{i} \PY{o+ow}{in} \PY{n+nb}{range}\PY{p}{(}\PY{n}{num\PYZus{}nodes}\PY{p}{)}\PY{p}{:}
                  \PY{n}{trace} \PY{o}{+}\PY{o}{=} \PY{n}{adj\PYZus{}matrix}\PY{p}{[}\PY{n}{i}\PY{p}{]}\PY{p}{[}\PY{n}{i}\PY{p}{]}
              \PY{k}{return} \PY{n}{trace}
\end{Verbatim}


    \begin{Verbatim}[commandchars=\\\{\}]
{\color{incolor}In [{\color{incolor}138}]:} \PY{k}{def} \PY{n+nf}{compute\PYZus{}num\PYZus{}of\PYZus{}triangles}\PY{p}{(}\PY{n}{G}\PY{p}{)}\PY{p}{:}
              \PY{n}{adj\PYZus{}matrix} \PY{o}{=} \PY{n}{nx}\PY{o}{.}\PY{n}{adjacency\PYZus{}matrix}\PY{p}{(}\PY{n}{G}\PY{p}{)}\PY{o}{.}\PY{n}{toarray}\PY{p}{(}\PY{p}{)}
              \PY{n}{num\PYZus{}nodes} \PY{o}{=} \PY{n}{G}\PY{o}{.}\PY{n}{number\PYZus{}of\PYZus{}nodes}\PY{p}{(}\PY{p}{)}
              
              \PY{c+c1}{\PYZsh{} To Store graph\PYZca{}2  }
              \PY{n}{graphv2} \PY{o}{=} \PY{p}{[}\PY{p}{[}\PY{k+kc}{None}\PY{p}{]} \PY{o}{*} \PY{n}{num\PYZus{}nodes} \PY{k}{for} \PY{n}{i} \PY{o+ow}{in} \PY{n+nb}{range}\PY{p}{(}\PY{n}{num\PYZus{}nodes}\PY{p}{)}\PY{p}{]} 
              \PY{c+c1}{\PYZsh{} To Store graph\PYZca{}3  }
              \PY{n}{graphv3} \PY{o}{=} \PY{p}{[}\PY{p}{[}\PY{k+kc}{None}\PY{p}{]} \PY{o}{*} \PY{n}{num\PYZus{}nodes} \PY{k}{for} \PY{n}{i} \PY{o+ow}{in} \PY{n+nb}{range}\PY{p}{(}\PY{n}{num\PYZus{}nodes}\PY{p}{)}\PY{p}{]} 
          
              \PY{c+c1}{\PYZsh{} Initialize}
              \PY{k}{for} \PY{n}{r} \PY{o+ow}{in} \PY{n+nb}{range}\PY{p}{(}\PY{n}{num\PYZus{}nodes}\PY{p}{)}\PY{p}{:}
                  \PY{k}{for} \PY{n}{c} \PY{o+ow}{in} \PY{n+nb}{range}\PY{p}{(}\PY{n}{num\PYZus{}nodes}\PY{p}{)}\PY{p}{:}
                      \PY{n}{graphv2}\PY{p}{[}\PY{n}{r}\PY{p}{]}\PY{p}{[}\PY{n}{c}\PY{p}{]} \PY{o}{=} \PY{n}{graphv3}\PY{p}{[}\PY{n}{r}\PY{p}{]}\PY{p}{[}\PY{n}{c}\PY{p}{]} \PY{o}{=} \PY{l+m+mi}{0}
            
              \PY{n}{graphv2} \PY{o}{=} \PY{n}{matrix\PYZus{}multiple}\PY{p}{(}\PY{n}{G}\PY{p}{,}\PY{n}{adj\PYZus{}matrix}\PY{p}{,} \PY{n}{adj\PYZus{}matrix}\PY{p}{,} \PY{n}{graphv2}\PY{p}{)}
              \PY{n}{graphv3} \PY{o}{=} \PY{n}{matrix\PYZus{}multiple}\PY{p}{(}\PY{n}{G}\PY{p}{,}\PY{n}{adj\PYZus{}matrix}\PY{p}{,} \PY{n}{graphv2}\PY{p}{,} \PY{n}{graphv3}\PY{p}{)}
          
              \PY{n}{trace} \PY{o}{=} \PY{n}{find\PYZus{}trace}\PY{p}{(}\PY{n}{G}\PY{p}{,}\PY{n}{graphv3}\PY{p}{)}
              \PY{k}{return} \PY{n+nb}{int}\PY{p}{(}\PY{n}{trace}\PY{o}{/}\PY{l+m+mi}{6}\PY{p}{)}
\end{Verbatim}


    \begin{Verbatim}[commandchars=\\\{\}]
{\color{incolor}In [{\color{incolor}139}]:} \PY{n}{adj\PYZus{}matrix} \PY{o}{=} \PY{n}{nx}\PY{o}{.}\PY{n}{adjacency\PYZus{}matrix}\PY{p}{(}\PY{n}{G}\PY{p}{)}\PY{o}{.}\PY{n}{toarray}\PY{p}{(}\PY{p}{)}
          \PY{n+nb}{print}\PY{p}{(}\PY{n}{adj\PYZus{}matrix}\PY{p}{)}
          \PY{n}{compute\PYZus{}num\PYZus{}of\PYZus{}triangles}\PY{p}{(}\PY{n}{G}\PY{p}{)}
          \PY{n+nb}{print}\PY{p}{(}\PY{l+s+s2}{\PYZdq{}}\PY{l+s+s2}{Number of Triangles in Graph G: }\PY{l+s+s2}{\PYZdq{}}\PY{p}{,} \PY{n}{compute\PYZus{}num\PYZus{}of\PYZus{}triangles}\PY{p}{(}\PY{n}{G}\PY{p}{)}\PY{p}{)}
\end{Verbatim}


    \begin{Verbatim}[commandchars=\\\{\}]
[[0 1 0 {\ldots} 0 0 0]
 [1 0 0 {\ldots} 0 0 0]
 [0 0 0 {\ldots} 0 0 0]
 {\ldots}
 [0 0 0 {\ldots} 0 0 0]
 [0 0 0 {\ldots} 0 0 1]
 [0 0 0 {\ldots} 0 1 0]]
Number of Triangles in Graph G:  13

    \end{Verbatim}

    \textbf{13. {[}10 marks{]}} Write a function \texttt{all\_triangles}
which finds all of the triangles in a graph. Use your function to count
the number of triangles in your graph, and compare with your answer to
question 12. (The two answers should, of course, be the same.)

Note: You will need to use your function in the next question, so you
should think carefully about what kind of data structure you want it to
output.

    \begin{Verbatim}[commandchars=\\\{\}]
{\color{incolor}In [{\color{incolor}135}]:} \PY{n}{num\PYZus{}nodes} \PY{o}{=} \PY{n}{G}\PY{o}{.}\PY{n}{number\PYZus{}of\PYZus{}nodes}\PY{p}{(}\PY{p}{)}
          \PY{n}{count\PYZus{}triangles} \PY{o}{=} \PY{l+m+mi}{0}
          \PY{k}{for} \PY{n}{i} \PY{o+ow}{in} \PY{n+nb}{range}\PY{p}{(}\PY{n}{num\PYZus{}nodes}\PY{p}{)}\PY{p}{:}
              \PY{n}{count\PYZus{}triangles} \PY{o}{+}\PY{o}{=} \PY{n}{nx}\PY{o}{.}\PY{n}{triangles}\PY{p}{(}\PY{n}{G}\PY{p}{,} \PY{n}{nodes}\PY{o}{=}\PY{n}{i}\PY{p}{)}
          \PY{n}{count\PYZus{}triangles} \PY{o}{=} \PY{n+nb}{int}\PY{p}{(}\PY{n}{count\PYZus{}triangles}\PY{o}{/}\PY{l+m+mi}{3}\PY{p}{)}
          \PY{n+nb}{print}\PY{p}{(}\PY{n}{count\PYZus{}triangles}\PY{p}{)}
\end{Verbatim}


    \begin{Verbatim}[commandchars=\\\{\}]
13

    \end{Verbatim}

    \begin{Verbatim}[commandchars=\\\{\}]
{\color{incolor}In [{\color{incolor}150}]:} \PY{k}{def} \PY{n+nf}{all\PYZus{}triangles}\PY{p}{(}\PY{n}{G}\PY{p}{)}\PY{p}{:}
              \PY{n}{num\PYZus{}nodes} \PY{o}{=} \PY{n}{G}\PY{o}{.}\PY{n}{number\PYZus{}of\PYZus{}nodes}\PY{p}{(}\PY{p}{)}
              \PY{n}{adj\PYZus{}matrix} \PY{o}{=} \PY{n}{nx}\PY{o}{.}\PY{n}{adjacency\PYZus{}matrix}\PY{p}{(}\PY{n}{G}\PY{p}{)}\PY{o}{.}\PY{n}{toarray}\PY{p}{(}\PY{p}{)}
              \PY{n}{count\PYZus{}triangles} \PY{o}{=} \PY{l+m+mi}{0}
              \PY{n}{list\PYZus{}triangles} \PY{o}{=} \PY{p}{[}\PY{p}{]} 
              \PY{c+c1}{\PYZsh{} for i in range(num\PYZus{}nodes):}
                  \PY{c+c1}{\PYZsh{} count\PYZus{}triangles += nx.triangles(G, nodes=i)}
              \PY{c+c1}{\PYZsh{} count\PYZus{}triangles = int(count\PYZus{}triangles/3)}
              
              \PY{c+c1}{\PYZsh{} Consider every possible triangles in graph }
              \PY{k}{for} \PY{n}{i} \PY{o+ow}{in} \PY{n+nb}{range}\PY{p}{(}\PY{n}{num\PYZus{}nodes}\PY{p}{)}\PY{p}{:} 
                  \PY{k}{for} \PY{n}{j} \PY{o+ow}{in} \PY{n+nb}{range}\PY{p}{(}\PY{n}{i}\PY{o}{+}\PY{l+m+mi}{1}\PY{p}{,}\PY{n}{num\PYZus{}nodes}\PY{p}{)}\PY{p}{:} 
                      \PY{k}{for} \PY{n}{k} \PY{o+ow}{in} \PY{n+nb}{range}\PY{p}{(}\PY{n}{j}\PY{o}{+}\PY{l+m+mi}{1}\PY{p}{,}\PY{n}{num\PYZus{}nodes}\PY{p}{)}\PY{p}{:} 
                          \PY{c+c1}{\PYZsh{} check if it is a cycle}
                          \PY{k}{if}\PY{p}{(} \PY{n}{i} \PY{o}{!=} \PY{n}{j} \PY{o+ow}{and} \PY{n}{i} \PY{o}{!=} \PY{n}{k} \PY{o+ow}{and} \PY{n}{j} \PY{o}{!=} \PY{n}{k} \PY{o+ow}{and} \PY{n}{adj\PYZus{}matrix}\PY{p}{[}\PY{n}{i}\PY{p}{]}\PY{p}{[}\PY{n}{j}\PY{p}{]} \PY{o+ow}{and} \PY{n}{adj\PYZus{}matrix}\PY{p}{[}\PY{n}{j}\PY{p}{]}\PY{p}{[}\PY{n}{k}\PY{p}{]} \PY{o+ow}{and} \PY{n}{adj\PYZus{}matrix}\PY{p}{[}\PY{n}{k}\PY{p}{]}\PY{p}{[}\PY{n}{i}\PY{p}{]}\PY{p}{)}\PY{p}{:} 
                              \PY{n}{count\PYZus{}triangles} \PY{o}{+}\PY{o}{=} \PY{l+m+mi}{1}
                              \PY{n}{list\PYZus{}triangles}\PY{o}{.}\PY{n}{append}\PY{p}{(}\PY{p}{[}\PY{n}{i}\PY{p}{,}\PY{n}{j}\PY{p}{,}\PY{n}{k}\PY{p}{]}\PY{p}{)}
              \PY{k}{return} \PY{n}{count\PYZus{}triangles}\PY{p}{,} \PY{n}{list\PYZus{}triangles}
\end{Verbatim}


    \begin{Verbatim}[commandchars=\\\{\}]
{\color{incolor}In [{\color{incolor}151}]:} \PY{n}{all\PYZus{}triangles}\PY{p}{(}\PY{n}{G}\PY{p}{)}
\end{Verbatim}


\begin{Verbatim}[commandchars=\\\{\}]
{\color{outcolor}Out[{\color{outcolor}151}]:} (13,
           [[0, 14, 45],
            [0, 21, 24],
            [2, 21, 32],
            [2, 21, 41],
            [2, 32, 39],
            [5, 12, 38],
            [5, 18, 21],
            [18, 21, 24],
            [18, 21, 32],
            [21, 24, 28],
            [23, 24, 28],
            [24, 43, 48],
            [35, 47, 48]])
\end{Verbatim}
            
    \textbf{14. {[}10 marks{]}} Re-draw your graph from Part I once more, so
that all of its triangles are clearly visible. You should use one colour
for the edges that appear in at least one triangle, and a different
colour for all other edges.

    \begin{Verbatim}[commandchars=\\\{\}]
{\color{incolor}In [{\color{incolor}197}]:} \PY{n}{triangle\PYZus{}list} \PY{o}{=} \PY{n}{all\PYZus{}triangles}\PY{p}{(}\PY{n}{G}\PY{p}{)}\PY{p}{[}\PY{l+m+mi}{1}\PY{p}{]}
          \PY{n+nb}{print}\PY{p}{(}\PY{n}{triangle\PYZus{}list}\PY{p}{)}
          \PY{n}{edges} \PY{o}{=} \PY{p}{[}\PY{p}{]}
          \PY{k}{for} \PY{n}{i} \PY{o+ow}{in} \PY{n+nb}{range}\PY{p}{(}\PY{n+nb}{len}\PY{p}{(}\PY{n}{triangle\PYZus{}list}\PY{p}{)}\PY{p}{)}\PY{p}{:}
              \PY{n}{edges}\PY{o}{.}\PY{n}{append}\PY{p}{(}\PY{p}{[}\PY{n}{triangle\PYZus{}list}\PY{p}{[}\PY{n}{i}\PY{p}{]}\PY{p}{[}\PY{l+m+mi}{0}\PY{p}{]}\PY{p}{,} \PY{n}{triangle\PYZus{}list}\PY{p}{[}\PY{n}{i}\PY{p}{]}\PY{p}{[}\PY{l+m+mi}{1}\PY{p}{]}\PY{p}{]}\PY{p}{)}
              \PY{n}{edges}\PY{o}{.}\PY{n}{append}\PY{p}{(}\PY{p}{[}\PY{n}{triangle\PYZus{}list}\PY{p}{[}\PY{n}{i}\PY{p}{]}\PY{p}{[}\PY{l+m+mi}{1}\PY{p}{]}\PY{p}{,} \PY{n}{triangle\PYZus{}list}\PY{p}{[}\PY{n}{i}\PY{p}{]}\PY{p}{[}\PY{l+m+mi}{2}\PY{p}{]}\PY{p}{]}\PY{p}{)}
              \PY{n}{edges}\PY{o}{.}\PY{n}{append}\PY{p}{(}\PY{p}{[}\PY{n}{triangle\PYZus{}list}\PY{p}{[}\PY{n}{i}\PY{p}{]}\PY{p}{[}\PY{l+m+mi}{2}\PY{p}{]}\PY{p}{,} \PY{n}{triangle\PYZus{}list}\PY{p}{[}\PY{n}{i}\PY{p}{]}\PY{p}{[}\PY{l+m+mi}{0}\PY{p}{]}\PY{p}{]}\PY{p}{)}
          \PY{n+nb}{print}\PY{p}{(}\PY{n}{edges}\PY{p}{)}
          
          \PY{n}{pos} \PY{o}{=} \PY{n}{nx}\PY{o}{.}\PY{n}{spring\PYZus{}layout}\PY{p}{(}\PY{n}{G}\PY{p}{)}
          \PY{n}{plt}\PY{o}{.}\PY{n}{figure}\PY{p}{(}\PY{n}{figsize}\PY{o}{=}\PY{p}{(}\PY{l+m+mi}{18}\PY{p}{,}\PY{l+m+mi}{18}\PY{p}{)}\PY{p}{)}
          \PY{n}{nx}\PY{o}{.}\PY{n}{draw\PYZus{}networkx}\PY{p}{(}\PY{n}{G}\PY{p}{,} \PY{n}{pos}\PY{p}{,} \PY{n}{node\PYZus{}color} \PY{o}{=} \PY{l+s+s1}{\PYZsq{}}\PY{l+s+s1}{Coral}\PY{l+s+s1}{\PYZsq{}}\PY{p}{,} \PY{n}{edge\PYZus{}color} \PY{o}{=} \PY{l+s+s1}{\PYZsq{}}\PY{l+s+s1}{black}\PY{l+s+s1}{\PYZsq{}}\PY{p}{,} \PY{n}{with\PYZus{}labels}\PY{o}{=}\PY{k+kc}{True}\PY{p}{)}
          \PY{n}{nx}\PY{o}{.}\PY{n}{draw\PYZus{}networkx\PYZus{}edges}\PY{p}{(}\PY{n}{G}\PY{p}{,}\PY{n}{pos}\PY{p}{,}\PY{n}{edgelist}\PY{o}{=}\PY{n}{edges}\PY{p}{,}\PY{n}{edge\PYZus{}color}\PY{o}{=}\PY{l+s+s1}{\PYZsq{}}\PY{l+s+s1}{RoyalBlue}\PY{l+s+s1}{\PYZsq{}}\PY{p}{,}\PY{n}{width}\PY{o}{=}\PY{l+m+mi}{5}\PY{p}{)}
          \PY{n}{nx}\PY{o}{.}\PY{n}{draw\PYZus{}networkx\PYZus{}labels}\PY{p}{(}\PY{n}{G}\PY{p}{,} \PY{n}{pos}\PY{p}{,} \PY{n}{font\PYZus{}size}\PY{o}{=}\PY{l+m+mi}{10}\PY{p}{,} \PY{n}{font\PYZus{}family}\PY{o}{=}\PY{l+s+s1}{\PYZsq{}}\PY{l+s+s1}{sans\PYZhy{}serif}\PY{l+s+s1}{\PYZsq{}}\PY{p}{)}
          \PY{n}{plt}\PY{o}{.}\PY{n}{xticks}\PY{p}{(}\PY{p}{[}\PY{p}{]}\PY{p}{)}
          \PY{n}{plt}\PY{o}{.}\PY{n}{yticks}\PY{p}{(}\PY{p}{[}\PY{p}{]}\PY{p}{)}
          \PY{n}{plt}\PY{o}{.}\PY{n}{show}\PY{p}{(}\PY{p}{)}
\end{Verbatim}


    \begin{Verbatim}[commandchars=\\\{\}]
[[0, 14, 45], [0, 21, 24], [2, 21, 32], [2, 21, 41], [2, 32, 39], [5, 12, 38], [5, 18, 21], [18, 21, 24], [18, 21, 32], [21, 24, 28], [23, 24, 28], [24, 43, 48], [35, 47, 48]]
[[0, 14], [14, 45], [45, 0], [0, 21], [21, 24], [24, 0], [2, 21], [21, 32], [32, 2], [2, 21], [21, 41], [41, 2], [2, 32], [32, 39], [39, 2], [5, 12], [12, 38], [38, 5], [5, 18], [18, 21], [21, 5], [18, 21], [21, 24], [24, 18], [18, 21], [21, 32], [32, 18], [21, 24], [24, 28], [28, 21], [23, 24], [24, 28], [28, 23], [24, 43], [43, 48], [48, 24], [35, 47], [47, 48], [48, 35]]

    \end{Verbatim}

    \begin{center}
    \adjustimage{max size={0.9\linewidth}{0.9\paperheight}}{output_72_1.png}
    \end{center}
    { \hspace*{\fill} \\}
    

    % Add a bibliography block to the postdoc
    
    
    
    \end{document}
